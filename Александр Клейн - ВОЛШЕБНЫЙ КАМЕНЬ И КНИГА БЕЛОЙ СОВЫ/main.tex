
\documentclass[oneside,final,14pt]{extreport}
%\usepackage[koi8-r]{inputenc}
\usepackage[russianb]{babel}
\usepackage{vmargin}
\setpapersize{A4}
\usepackage[T2A]{fontenc}
\usepackage[utf8x]{inputenc}  % more recent versions (at least>=2004-17-10)
%\usepackage[russian]{babel}
\setmarginsrb{2cm}{1.5cm}{1cm}{1.5cm}{0pt}{0mm}{0pt}{13mm}
\usepackage{indentfirst}
\usepackage{nicefrac} % For comparison
%\usepackage{xfrac}    % Works better with other fonts
\usepackage[unicode, pdftex]{hyperref}
\usepackage{lettrine}
\usepackage[usenames]{color}
\usepackage{colortbl}
\sloppy
\begin{document}
	
	\section*{Описание}
	
	{\bf Название:} Волшебный Камень и Книга Белой Совы
	
	{\bf Автор:} Александр Соломонович Клейн
	
	{\bf Издательство:} Сыктывкар: Коми книжное издательство, 1986
	
	{\bf Художник:} Г. С. Тонков
	
	{\bf Редактор:} А. В. Некрасов
	
	{\bf Художественный редактор:} В. Б. Осипов
	
	{\bf Технический редактор:} А. Н. Вишнева
	
	{\bf Корректоры:} А. А. Надуткина, Т. И. Попова
	
	{\bf Аннотация:} Эта повесть-сказка, написанная на основе древних коми поверий и преданий, посвящена удивительным приключениям добрых и смелых людей, их борьбе против злых сил за право владеть волшебным камнем и Книгой мудрости.
	\thispagestyle{empty} % выключаем отображение номера для этой страницы
	\newpage
	
	\begin{titlepage}
		
		\begin{center}
			%\vfill
			
			%\vfill
			\topskip0pt
			\vspace*{\fill}
			
			
			{\large\bf Александр Клейн\\}
			\ \\
			\ \\
			{\Huge\bf ВОЛШЕБНЫЙ КАМЕНЬ И КНИГА БЕЛОЙ СОВЫ\\}
			\ \\
			\ \\
			\ \\
			Повесть-сказка
			\ \\
			\ \\
			По мотивам коми фольклора
			\vspace*{\fill}    
			
			\vfill
			
			Сыктывкар
			
			Коми книжное издательство 1986
		\end{center}
		
	\end{titlepage}
	
	\thispagestyle{empty} % выключаем отображение номера для этой страницы
	
	\newpage
	
	\begin{flushleft}
		\begin{verse}
			\qquad \qquad Читатель-друг, ступай со мной\\
			\qquad \qquad Туда, где лес стоит стеной\\
			\qquad \qquad И птицы людям говорят\\
			\qquad \qquad Про вьюжный край, где спрятан клад,\\
			\qquad \qquad Где серебром блестят равнины,\\
			\qquad \qquad А в горных облачных вершинах\\
			\qquad \qquad Гнездятся ветры.\\
			\qquad \qquad\qquad\qquad\qquad\qquad Труден путь.\\
			\qquad \qquad Но не робей и рядом будь.\\
			\qquad \qquad Знай: одному — и мне беда.\\
			\qquad \qquad С тобой — согласен хоть куда;\\
			\qquad \qquad Один — тяжел я на подъём.\\
			\qquad \qquad А вместе — в сказку мы войдём!\\
		\end{verse}
	\end{flushleft}
	
	{%
		\centering
		\subsection*{\textcolor{red}{1. Как исчезло Зелёное Царство}}
	}
	
	\lettrine[findent=0pt]{\textbf{\textcolor{red}{К}}}{} аждые сто лет в Книге мудрости прибавлялся только один лист. Никто не мог объяснить, как это происходило, но с каждым столетием Книга становилась чуть толще. Она была написана не совсем обычно, говорят, даже стихами. В ней было много полезных советов. Но никто из людей её никогда не видел. А о её существовании знали давно. Вероятно, старые охотники, хорошо понимающие язык птиц, услышали о чудесной Книге от них. Птицы не умеют хранить тайны, особенно сороки и воробьи. Они крошки не клюнут без того, чтобы не пискнуть, не сказать что-нибудь на своём языке, не посплетничать.
	
	А птицы, безусловно, должны были знать о Книге мудрости, потому что её хранительницей была тоже птица — Белая полярная Сова. Вы её видели когда-нибудь? Огромная, круглоголовая, с большими жёлто-огненными глазами и загнутым книзу, наподобие крючка, острым клювом; с могучими широкими крыльями, покрытыми маленькими коричневыми черточками и крапинками. У неё очень тонкий слух и отличное чутье. Неправду говорят, будто Белая Сова хорошо видит только ночью. Нет, она великолепно видит и днем. Ведь там, где она обитает, в бескрайних северных равнинах, половину года царит день, а ночи в это время вообще нет. Ночь там только зимой.
	
	Почётная должность хранительницы Книги мудрости передавалась совами из рода в род. Никто не помнит — когда, сколько тысяч лет тому назад пра-пра-пра-пра-пра-прабабушка одной Белой Совы, — а совы минут очень долго, — вложила в золотистый переплёт из непромокаемой нерпичьей шкуры первый лист заветной Книги. Никто не видел и того, кто сделал этот чудесный переплёт. Но... так уж случилось, что когда-то, давным-давно, бессмертные властители сурового и прекрасного севера, снежная буря Пурга и злой ветер Хановей, сами не умея читать, доверили белым совам хранение Книги мудрости.
	
	Раз в сто лет Хановей и Пурга интересовались тем, что в ней написано, и Сова читала им вслух о богатствах подвластной им земли, о людях, которые живут в этих холодных краях.
	
	После чтения Сова уносила Книгу в какое-нибудь потайное место, а Хановей и Пурга от безделия снова начинали кружить по просторам, заносить снегом стойбища оленеводов, морозить и сбивать с дороги одиноких путников. Наподобие исполинских чумов вздымались сугробы, оглушительно свистел ветер, снежные вихри закрывали небо. Тяжело приходилось тому, кого настигали Хановей и Пурга среди Великой северной равнины, тундры.
	
	Не всегда она была такой безжизненной. На месте болотных кочек, ложбин и голых холмов, покрытых чахлым кустарником, когда-то буйно зеленели травы, высились могучие деревья; на сказочно красивых цветах, как на волнах, пестрея крыльями, покачивались огромные бабочки. Прозрачный воздух был напоён душистой свежестью и, казалось, даже слегка вздрагивал от птичьих трелей.
	
	А посреди этого зелёного царства искрился и сиял Камень, несравненно прекраснее и дороже всех драгоценных камней — волшебный Камень жизни. Его щедрые лучи оживляли все вокруг, несли тепло и свет, радость и счастье. И под пение птиц мирно дремали чудо-леса в лучах доброго Камня.
	
	Очень давно это было... Так давно, что ученые люди до сих пор все считают и никак не могут точно сосчитать, сколько лет прошло с тех пор. Не могут они точно назвать не то что день, а даже месяц или год, когда впервые прилетели в этот край Хановей и его сестра Пурга. Они были такие же, как сейчас, лохматые, шумливые, неряшливые. Седая косматая Пурга и тогда была старухой. Она, говорят. сразу родилась старой. Да и братец ее Хановей был не моложе.
	
	Как и откуда они залетели? Неизвестно. Во всяком случае, не из жарких мест, потому что оба терпеть не могли, не выносили ласкового тепла и света. Но увидели они окружающую красоту, услышали пение птиц и невольно замерли. Уж на что был груб и неотёсан Хановей, и тот не выдержал: глубоко вздохнул, залюбовался и на его сосулистую бороду, дрожа, закапали слезинки умиления.
	
	Но Пурга быстро вывела брата из оцепенения.
	
	— Чего уставился, бородач? — прохрипела она. — С твоих усов скоро ручьи потекут и мокрое место от тебя останется, а ты рот разинул. Ведь этакая жара нас вот-вот погубит. Я задыхаюсь от запаха цветов и глохну от птичьего гомона. Пора положить конец безобразию.
	
	— Смотри!-— она указала на сверкающий Камень жизни. — Вот виновник всей этой кутерьмы. Его нужно убрать. Слышишь?! Только тогда мы сможем свободно разгуливать по земле. Ну-ка, принимайся за дело.
	
	С этими словами злая старуха дико завыла, затрясла косматой головой — и на заветный Камень, на зеленые деревья, на цветы повалил снег.
	
	Хановей не отстал от сестрицы: он засвистел по-разбойничьи и, откуда ни возьмись, на цветущую землю обрушился страшный ураган. Он ломал кусты и деревья, вдавливал их глубоко в землю, и она, еще недавно такая теплая и податливая, становилась твердой, как лёд, который, подчиняясь приказу Пурги и Хановея, лавинами наступал на зеленое царство.
	
	Все свои силы собрали Хановей и Пурга против Камня жизни. Долго его чудесные лучи не позволяли врагам приблизиться к нему, расплавляли снег и лед, пытались посылать живительное тепло окружающим лесам. Злые пришельцы отступили. Но ненадолго. Пурга исчезла лишь затем, чтобы подкрасться сзади, со стороны Большого холодного моря и окатить Камень жизни гигантскими волнами. А Хановей взобрался на высокие горы, дохнул во всю мощь морозным ветром, и волны, нахлынув на Камень, мгновенно застыли, сковали его толстым льдом и вдавили глубоко в землю.
	
	Мигнул и исчез последний луч... Загоготали от радости злые силы, заплясали над холмами, покрывшими волшебный Камень. От прекрасных лесов остались только редкие кусты да согнутые в три погибели карликовые берёзки.
	
	Так воцарились на севере Хановей и Пурга. Седые, смертельно бледные, они больше всего любили белый цвет, а потому щедро осыпали свои владения снегом. И под цвет снега, боясь грозных хозяев, стали рядиться и зайцы, и песцы, и куропатки, и другие звери и птицы, одним словом, почти все жители бескрайной северной равнины, тундры.
	
	\
	{%
		\centering
		\subsection*{\textcolor{red}{2. УСЛОВНЫЙ ЗНАК}}
	}
	
	\lettrine[findent=0pt]{\textbf{\textcolor{red}{П}}}{} рошло много тысяч лет...
	
\qquad	Чувствуя себя безраздельными хозяевами, Пурга и Хановей стали беспечнее, обленились. Хановей завел себе быстрейших оленей, и стремительно мчали они его, куда ему заблагорассудится. Порой он сажал в свои серебряные сани-нарты Пургу, и катались они вдвоем вволю.
	
	Но, хотя свободного времени хватало, учиться они ничему не захотели, даже грамоте. Зачем она им. когда у них есть Белая Сова? Та и читать и даже считать умеет.
	
	Как-то накануне окончания длинной зимней ночи Пурга заметила в нечёсаной бороде Хановея совиное перо с тремя темными точками. Это был условный знак.
	
	— Белая Сова вызывает нас. — сказала Пурга, осмотрев перо. — Значит, завтра исполнится ровно сто лет с того дня, как мы в последний раз заглядывали в Книгу мудрости. Пора опять заглянуть.
	
	— А что там может быть нового? — буркнул Ханоней. — Нас-то она не касается.
	
	— Как знать? — покачала головой Пурга. — Книга — такая вещь, от которой всегда можно ожидать неожиданное, даже нам. Отправимся к Сове. Посмотрим, что она делает.
	
	\
	{%
		\centering
		\subsection*{\textcolor{red}{3. НОВЫЕ ЗНАКОМЫЕ}}
	}
	
	\lettrine[findent=0pt]{\textbf{\textcolor{red}{К}}}{} азалось, Белая Сова в то утра надела королевскую горностаевую мантию. Все пёрышки были приглажены, вычищены до блеска. По виду никто бы не подумал, что красавице уже за двести лет: она выглядела, по крайней мерь, в два раза моложе. Время от времени в её больших круглых глазах вспыхивали язычки жёлтого пламени: отблески лучей далекого солнца. Оно перед зимней спячкой хотело проститься с тундрой.
	
	Сова сидела на лохматой кочке. Позади начинались заросли кустарника. Из него робко выглядывали посиневшие от ранних морозов горбатые ивы и костлявые берёзки.
	
	Но вот над гребнем одного из холмов появилась узкая багрово-красная лента: краешек солнца. Оно медленно, с трудом приподнялось над снежной пустыней. В то же мгновение Сова почувствовала под левым крылом чуть заметный толчок: в Книге мудрости прибавился еще один лист.
	
	Сова бережно вынула большую теплую Книгу из-под крыла, положила перед собой на заранее расчищенное место, на другую высокую кочку, и раскрыла.
	
	Сейчас лучи солнца озарят свежую, еще чистую страницу, и на ней выступят буквы. Тогда можно будет узнать, что вошло в вечную Книгу за сто лет.
	
	Сова поглядела по сторонам и склонилась над страницей. Страница слегка потемнела. Как сквозь туман, на ней проступали очертания отдельных букв. А Пурги и Хановея все не было.
	
	— Плохо, когда хозяева такие невоспитанные и неаккуратные, а доверить великую тайну предков некому, — горестно подумала Сова, и вдруг услышала, как за её спиной сухо треснули мерзлые ветки кустарника.
	
	— Доверь мне тайну своих предков. — словно читая её мысли, просипел над ухом птицы простуженный голос.— ‘Только доверь — и я открою тебе двери моего вечнозеленого царства.
	
	Сова стремительно расправила крылья над Книгой мудрости и резко повернула круглую голову назад.
	
	Там стояло странное горбатое существо — то ли человек, то ли зверь, то ли дерево?.. Из-под огромной зелёной шапки-шишки торчали мохнатые уши; посреди лба одиноко блестел круглый глаз; могучие плечи переходили в длинные руки. Они сжимали толстую дубину. На кривых ногах, одна из которых напоминала копыто, виднелись бесчисленные бородавки.
	
	— Ну, чего смотришь? Отдай! Не пожалеешь, — настойчиво повторил незнакомец.
	
	— Отдай мне, — тут же прошипел другой голое, и из-за кустов вышла скрюченная длинноносая старуха. Одета она была в пестрые лохмотья, а на её голове кокетливо покачивался щербатый глиняный горшок.
	
	— Отдай мне, — вкрадчиво продолжала старуха и протянула жилистую руку к Книге. — И я до скончания дней твоих дам тебе угол и пищу в жаркой избе.
	
	Одноглазый резко выпрямился:
	
	— Прочь!
	
	Старуха отпрянула в сторону и споткнулась. Горшок упал с её головы и разбился.
	
	— Тьфу, Вэрса безмозглый! — выругалась она. — Да зачем тебе Книга мудрости? Ты же глуп, как самый последний черепок этого разбитого горшка. Отдай мне Книгу. — прогнусавила она, приближаясь сбоку к Сове.
	
	— Нет, мне! — зарычал Вэрса. — Уйди, Баба-Ёма! Уйди!
	
	Но грозно поднятая дубина вдруг вырвалась из крючковатых пальцев Вэрсы и отлетела далеко в сторону.
	
	— Во-он! Во-он! — внезапно завыло и засвистело вокруг. — Во-он отсюда! Мы — хозяева Великой северной равнины! Во-он!..
	
	Откуда ни возьмись, налетел страшный вихрь. Снежная вьюга в одно мгновение ослепила пришельцев. Бешеный ветер сбил их с ног, закрутил, приподнял и понес прочь в далекие темные леса.
	
	Пурга и Хановей подоспели к Сове. Она все еще лежала на Книге, прикрывая её всем телом. Она поднялась только тогда, когда хозяева прогнали Вэрсу и Бабу-Ёму.
	
	
\end{document}







