
\documentclass[oneside,final,14pt]{extreport}
%\usepackage[koi8-r]{inputenc}
\usepackage[russianb]{babel}
\usepackage{vmargin}
\setpapersize{A4}
\usepackage[T2A]{fontenc}
\usepackage[utf8x]{inputenc}  % more recent versions (at least>=2004-17-10)
%\usepackage[russian]{babel}
\setmarginsrb{2cm}{1.5cm}{1cm}{1.5cm}{0pt}{0mm}{0pt}{13mm}
\usepackage{indentfirst}
\usepackage{nicefrac} % For comparison
%\usepackage{xfrac}    % Works better with other fonts
\usepackage[unicode, pdftex]{hyperref}
\usepackage{lettrine}
\usepackage[usenames]{color}
\usepackage{colortbl}
\sloppy
\begin{document}
	
	\section*{Описание}
	
	{\bf Название:} Волшебный Камень и Книга Белой Совы
	
	{\bf Автор:} Александр Соломонович Клейн
	
	{\bf Издательство:} Сыктывкар: Коми книжное издательство, 1986
	
	{\bf Художник:} Г. С. Тонков
	
	{\bf Редактор:} А. В. Некрасов
	
	{\bf Художественный редактор:} В. Б. Осипов
	
	{\bf Технический редактор:} А. Н. Вишнева
	
	{\bf Корректоры:} А. А. Надуткина, Т. И. Попова
	
	{\bf Аннотация:} Эта повесть-сказка, написанная на основе древних коми поверий и преданий, посвящена удивительным приключениям добрых и смелых людей, их борьбе против злых сил за право владеть волшебным камнем и Книгой мудрости.
	\thispagestyle{empty} % выключаем отображение номера для этой страницы
	\newpage
	
	\begin{titlepage}
		
		\begin{center}
			%\vfill
			
			%\vfill
			\topskip0pt
			\vspace*{\fill}
			
			
			{\large\bf Александр Клейн\\}
			\ \\
			\ \\
			{\Huge\bf ВОЛШЕБНЫЙ КАМЕНЬ И КНИГА БЕЛОЙ СОВЫ\\}
			\ \\
			\ \\
			\ \\
			Повесть-сказка
			\ \\
			\ \\
			По мотивам коми фольклора
			\vspace*{\fill}    
			
			\vfill
			
			Сыктывкар
			
			Коми книжное издательство 1986
		\end{center}
		
	\end{titlepage}
	
	\thispagestyle{empty} % выключаем отображение номера для этой страницы
	
	\newpage
	
	\begin{flushleft}
		\begin{verse}
			\qquad \qquad Читатель-друг, ступай со мной\\
			\qquad \qquad Туда, где лес стоит стеной\\
			\qquad \qquad И птицы людям говорят\\
			\qquad \qquad Про вьюжный край, где спрятан клад,\\
			\qquad \qquad Где серебром блестят равнины,\\
			\qquad \qquad А в горных облачных вершинах\\
			\qquad \qquad Гнездятся ветры.\\
			\qquad \qquad\qquad\qquad\qquad\qquad Труден путь.\\
			\qquad \qquad Но не робей и рядом будь.\\
			\qquad \qquad Знай: одному — и мне беда.\\
			\qquad \qquad С тобой — согласен хоть куда;\\
			\qquad \qquad Один — тяжел я на подъём.\\
			\qquad \qquad А вместе — в сказку мы войдём!\\
		\end{verse}
	\end{flushleft}
	
	{%
		\centering
		\subsection*{\textcolor{red}{1. КАК ИСЧЕЗЛО ЗЕЛЁНОЕ ЦАРСТВО}}
	}
	
	\lettrine[findent=0pt]{\textbf{\textcolor{red}{К}}}{} аждые сто лет в Книге мудрости прибавлялся только один лист. Никто не мог объяснить, как это происходило, но с каждым столетием Книга становилась чуть толще. Она была написана не совсем обычно, говорят, даже стихами. В ней было много полезных советов. Но никто из людей её никогда не видел. А о её существовании знали давно. Вероятно, старые охотники, хорошо понимающие язык птиц, услышали о чудесной Книге от них. Птицы не умеют хранить тайны, особенно сороки и воробьи. Они крошки не клюнут без того, чтобы не пискнуть, не сказать что-нибудь на своём языке, не посплетничать.
	
	А птицы, безусловно, должны были знать о Книге мудрости, потому что её хранительницей была тоже птица — Белая полярная Сова. Вы её видели когда-нибудь? Огромная, круглоголовая, с большими жёлто-огненными глазами и загнутым книзу, наподобие крючка, острым клювом; с могучими широкими крыльями, покрытыми маленькими коричневыми черточками и крапинками. У неё очень тонкий слух и отличное чутье. Неправду говорят, будто Белая Сова хорошо видит только ночью. Нет, она великолепно видит и днем. Ведь там, где она обитает, в бескрайних северных равнинах, половину года царит день, а ночи в это время вообще нет. Ночь там только зимой.
	
	Почётная должность хранительницы Книги мудрости передавалась совами из рода в род. Никто не помнит — когда, сколько тысяч лет тому назад пра-пра-пра-пра-пра-прабабушка одной Белой Совы, — а совы минут очень долго, — вложила в золотистый переплёт из непромокаемой нерпичьей шкуры первый лист заветной Книги. Никто не видел и того, кто сделал этот чудесный переплёт. Но... так уж случилось, что когда-то, давным-давно, бессмертные властители сурового и прекрасного севера, снежная буря Пурга и злой ветер Хановей, сами не умея читать, доверили белым совам хранение Книги мудрости.
	
	Раз в сто лет Хановей и Пурга интересовались тем, что в ней написано, и Сова читала им вслух о богатствах подвластной им земли, о людях, которые живут в этих холодных краях.
	
	После чтения Сова уносила Книгу в какое-нибудь потайное место, а Хановей и Пурга от безделия снова начинали кружить по просторам, заносить снегом стойбища оленеводов, морозить и сбивать с дороги одиноких путников. Наподобие исполинских чумов вздымались сугробы, оглушительно свистел ветер, снежные вихри закрывали небо. Тяжело приходилось тому, кого настигали Хановей и Пурга среди Великой северной равнины, тундры.
	
	Не всегда она была такой безжизненной. На месте болотных кочек, ложбин и голых холмов, покрытых чахлым кустарником, когда-то буйно зеленели травы, высились могучие деревья; на сказочно красивых цветах, как на волнах, пестрея крыльями, покачивались огромные бабочки. Прозрачный воздух был напоён душистой свежестью и, казалось, даже слегка вздрагивал от птичьих трелей.
	
	А посреди этого зелёного царства искрился и сиял Камень, несравненно прекраснее и дороже всех драгоценных камней — волшебный Камень жизни. Его щедрые лучи оживляли все вокруг, несли тепло и свет, радость и счастье. И под пение птиц мирно дремали чудо-леса в лучах доброго Камня.
	
	Очень давно это было... Так давно, что ученые люди до сих пор все считают и никак не могут точно сосчитать, сколько лет прошло с тех пор. Не могут они точно назвать не то что день, а даже месяц или год, когда впервые прилетели в этот край Хановей и его сестра Пурга. Они были такие же, как сейчас, лохматые, шумливые, неряшливые. Седая косматая Пурга и тогда была старухой. Она, говорят. сразу родилась старой. Да и братец ее Хановей был не моложе.
	
	Как и откуда они залетели? Неизвестно. Во всяком случае, не из жарких мест, потому что оба терпеть не могли, не выносили ласкового тепла и света. Но увидели они окружающую красоту, услышали пение птиц и невольно замерли. Уж на что был груб и неотёсан Хановей, и тот не выдержал: глубоко вздохнул, залюбовался и на его сосулистую бороду, дрожа, закапали слезинки умиления.
	
	Но Пурга быстро вывела брата из оцепенения.
	
	— Чего уставился, бородач? — прохрипела она. — С твоих усов скоро ручьи потекут и мокрое место от тебя останется, а ты рот разинул. Ведь этакая жара нас вот-вот погубит. Я задыхаюсь от запаха цветов и глохну от птичьего гомона. Пора положить конец безобразию.
	
	— Смотри!-— она указала на сверкающий Камень жизни. — Вот виновник всей этой кутерьмы. Его нужно убрать. Слышишь?! Только тогда мы сможем свободно разгуливать по земле. Ну-ка, принимайся за дело.
	
	С этими словами злая старуха дико завыла, затрясла косматой головой — и на заветный Камень, на зеленые деревья, на цветы повалил снег.
	
	Хановей не отстал от сестрицы: он засвистел по-разбойничьи и, откуда ни возьмись, на цветущую землю обрушился страшный ураган. Он ломал кусты и деревья, вдавливал их глубоко в землю, и она, еще недавно такая теплая и податливая, становилась твердой, как лёд, который, подчиняясь приказу Пурги и Хановея, лавинами наступал на зеленое царство.
	
	Все свои силы собрали Хановей и Пурга против Камня жизни. Долго его чудесные лучи не позволяли врагам приблизиться к нему, расплавляли снег и лед, пытались посылать живительное тепло окружающим лесам. Злые пришельцы отступили. Но ненадолго. Пурга исчезла лишь затем, чтобы подкрасться сзади, со стороны Большого холодного моря и окатить Камень жизни гигантскими волнами. А Хановей взобрался на высокие горы, дохнул во всю мощь морозным ветром, и волны, нахлынув на Камень, мгновенно застыли, сковали его толстым льдом и вдавили глубоко в землю.
	
	Мигнул и исчез последний луч... Загоготали от радости злые силы, заплясали над холмами, покрывшими волшебный Камень. От прекрасных лесов остались только редкие кусты да согнутые в три погибели карликовые берёзки.
	
	Так воцарились на севере Хановей и Пурга. Седые, смертельно бледные, они больше всего любили белый цвет, а потому щедро осыпали свои владения снегом. И под цвет снега, боясь грозных хозяев, стали рядиться и зайцы, и песцы, и куропатки, и другие звери и птицы, одним словом, почти все жители бескрайной северной равнины, тундры.
	
	\
	{%
		\centering
		\subsection*{\textcolor{red}{2. УСЛОВНЫЙ ЗНАК}}
	}
	
	\lettrine[findent=0pt]{\textbf{\textcolor{red}{П}}}{} рошло много тысяч лет...
	
	\qquad	Чувствуя себя безраздельными хозяевами, Пурга и Хановей стали беспечнее, обленились. Хановей завел себе быстрейших оленей, и стремительно мчали они его, куда ему заблагорассудится. Порой он сажал в свои серебряные сани-нарты Пургу, и катались они вдвоем вволю.
	
	Но, хотя свободного времени хватало, учиться они ничему не захотели, даже грамоте. Зачем она им. когда у них есть Белая Сова? Та и читать и даже считать умеет.
	
	Как-то накануне окончания длинной зимней ночи Пурга заметила в нечёсаной бороде Хановея совиное перо с тремя темными точками. Это был условный знак.
	
	— Белая Сова вызывает нас. — сказала Пурга, осмотрев перо. — Значит, завтра исполнится ровно сто лет с того дня, как мы в последний раз заглядывали в Книгу мудрости. Пора опять заглянуть.
	
	— А что там может быть нового? — буркнул Ханоней. — Нас-то она не касается.
	
	— Как знать? — покачала головой Пурга. — Книга — такая вещь, от которой всегда можно ожидать неожиданное, даже нам. Отправимся к Сове. Посмотрим, что она делает.
	
	\
	{%
		\centering
		\subsection*{\textcolor{red}{3. НОВЫЕ ЗНАКОМЫЕ}}
	}
	
	\lettrine[findent=0pt]{\textbf{\textcolor{red}{К}}}{} азалось, Белая Сова в то утра надела королевскую горностаевую мантию. Все пёрышки были приглажены, вычищены до блеска. По виду никто бы не подумал, что красавице уже за двести лет: она выглядела, по крайней мерь, в два раза моложе. Время от времени в её больших круглых глазах вспыхивали язычки жёлтого пламени: отблески лучей далекого солнца. Оно перед зимней спячкой хотело проститься с тундрой.
	
	Сова сидела на лохматой кочке. Позади начинались заросли кустарника. Из него робко выглядывали посиневшие от ранних морозов горбатые ивы и костлявые берёзки.
	
	Но вот над гребнем одного из холмов появилась узкая багрово-красная лента: краешек солнца. Оно медленно, с трудом приподнялось над снежной пустыней. В то же мгновение Сова почувствовала под левым крылом чуть заметный толчок: в Книге мудрости прибавился еще один лист.
	
	Сова бережно вынула большую теплую Книгу из-под крыла, положила перед собой на заранее расчищенное место, на другую высокую кочку, и раскрыла.
	
	Сейчас лучи солнца озарят свежую, еще чистую страницу, и на ней выступят буквы. Тогда можно будет узнать, что вошло в вечную Книгу за сто лет.
	
	Сова поглядела по сторонам и склонилась над страницей. Страница слегка потемнела. Как сквозь туман, на ней проступали очертания отдельных букв. А Пурги и Хановея все не было.
	
	— Плохо, когда хозяева такие невоспитанные и неаккуратные, а доверить великую тайну предков некому, — горестно подумала Сова, и вдруг услышала, как за её спиной сухо треснули мерзлые ветки кустарника.
	
	— Доверь мне тайну своих предков. — словно читая её мысли, просипел над ухом птицы простуженный голос.— ‘Только доверь — и я открою тебе двери моего вечнозеленого царства.
	
	Сова стремительно расправила крылья над Книгой мудрости и резко повернула круглую голову назад.
	
	Там стояло странное горбатое существо — то ли человек, то ли зверь, то ли дерево?.. Из-под огромной зелёной шапки-шишки торчали мохнатые уши; посреди лба одиноко блестел круглый глаз; могучие плечи переходили в длинные руки. Они сжимали толстую дубину. На кривых ногах, одна из которых напоминала копыто, виднелись бесчисленные бородавки.
	
	— Ну, чего смотришь? Отдай! Не пожалеешь, — настойчиво повторил незнакомец.
	
	— Отдай мне, — тут же прошипел другой голое, и из-за кустов вышла скрюченная длинноносая старуха. Одета она была в пестрые лохмотья, а на её голове кокетливо покачивался щербатый глиняный горшок.
	
	— Отдай мне, — вкрадчиво продолжала старуха и протянула жилистую руку к Книге. — И я до скончания дней твоих дам тебе угол и пищу в жаркой избе.
	
	Одноглазый резко выпрямился:
	
	— Прочь!
	
	Старуха отпрянула в сторону и споткнулась. Горшок упал с её головы и разбился.
	
	— Тьфу, Вэрса безмозглый! — выругалась она. — Да зачем тебе Книга мудрости? Ты же глуп, как самый последний черепок этого разбитого горшка. Отдай мне Книгу. — прогнусавила она, приближаясь сбоку к Сове.
	
	— Нет, мне! — зарычал Вэрса. — Уйди, Баба-Ёма! Уйди!
	
	Но грозно поднятая дубина вдруг вырвалась из крючковатых пальцев Вэрсы и отлетела далеко в сторону.
	
	— Во-он! Во-он! — внезапно завыло и засвистело вокруг. — Во-он отсюда! Мы — хозяева Великой северной равнины! Во-он!..
	
	Откуда ни возьмись, налетел страшный вихрь. Снежная вьюга в одно мгновение ослепила пришельцев. Бешеный ветер сбил их с ног, закрутил, приподнял и понес прочь в далекие темные леса.
	
	Пурга и Хановей подоспели к Сове. Она все еще лежала на Книге, прикрывая её всем телом. Она поднялась только тогда, когда хозяева прогнали Вэрсу и Бабу-Ёму.
	
	\
	{%
		\centering
		\subsection*{\textcolor{red}{4. ТРЕВОГА}}
	}
	
	\lettrine[findent=0pt]{\textbf{\textcolor{red}{Н}}}{} -н-н-у-у-у, — спросила Пурга, — что там у тебя?
	
	\qquad  	 — Вот, — Сова торжественно опустила правую ногу на открытую страницу. На ней сверкали позолотой свежие строчки. Каждая начиналась с крупной, очень красивой заглавной буквы.
	
	Хановей шумно вздохнул и посмотрел на сестру. Та равнодушно кивнула Сове:
	
	— Читай.
	
	Но первые же слова новой страницы встревожили хозяев севера.
	
	— Кто чудесным Камнем жизни,
	
	Страх отбросив, завладеет, — ровным голосом читала Сова.
	
	— Опять об этом негодном Камне! — возмущенно топнул ногой Хановей. — Да когда же?!.
	
	— Погоди, — перебила Пурга. — Слушай дальше.
	
	— Тот бесценный дар получит. — продолжала читать Сова.—
	
	И погубит злые силы,
	
	Что над тундрою царили.
	
	— Постой! — не выдержал Хановей. — Повтори.
	
	Сова исполнила желание властелина.
	
	— Что скажешь, братец? — тихо спросила Пурга.
	
	Хановей, сердито сопя, уткнулся ледяным носом в Книгу.
	
	— А-а-а?.. — вопросительно протянула Пурга. Хановей не ответил.
	
	— Давай-ка, подумаем, — прошамкала Пурга. — Как ни трудно, а думать иногда нужно. Помолчим и подумаем. Подумать стоит крепко...
	
	Все потупили головы и так просидели три дня и три ночи.
	
	Первый нарушил молчание Хановей.
	
	— Я придумал! — гаркнул он во все горло. — Надо уничтожить Книгу.
	
	Пурга так затряслась от беззвучного смеха, что с её седой головы во все стороны полетели снежные вихри.
	
	— Ох-хо-хо! — застонала она. — Ты же знаешь: этого сделать нельзя. Что попало в Книгу мудрости, то в огне не горит и в воде не тонет. Когда ты это поймешь?
	
	— А если разорвать Книгу?!
	
	При этих словах Сова с возмущением посмотрела на невежду, а Пурга печально заметила:
	
	— Что ж, разлетятся её страницы, и кому-нибудь попадет эта, — старуха кивнула на новый лист. — Люди найдут его и отправятся за Камнем жизни.
	
	— Но что же делать, что делать, что делать? — зачастил Хановей. — Мы же погибнем.
	
	— Пожалуй, следует поступить так, — отделяя слова, отчеканила Пурга. — Нужно подальше спрятать Книгу, чтобы никто не смог её найти. Белая Сова! — властно обратилась она к молчаливо следившей за своими хозяевами птице. — Отнеси Книгу мудрости каменное гнездо ветров, на непреступную вершину Тэв-Поз-Из!
	
	— На Тэв-Поз-Из! — как эхо, повторил Хановей. — На Тэв-Поз-Из!
	
	— Там, — продолжала злая старуха, — я прошью её страницы вечным холодом.
	
	—А я обрадовался Хановей, — скую их вечным льдом и не так-то просто будет найти Книгу и развернуть её листы. И. удивляясь собственной догадливости, бородач несколько раз подпрыгнул на месте.
	
	Чего смотришь? — вдруг накинулась Пурга на Сову. — Сказано: на Тэв-Поз-Из — и все тут. Лети!
	
	Сова решила про себя, что обязательно дочитает страницу в более спокойной обстановке, аккуратно закрыла Книгу, обвязала её мягким ремешком, перекинула через плечо и пустилась в путь.
	
	\
	{%
		\centering
		\subsection*{\textcolor{red}{5. КОЛДУНЬЯ И СЛЕПОЙ}}
	}
	
	\lettrine[findent=0pt]{\textbf{\textcolor{red}{Т}}}{}  еперь оставим на время Белую Сову и её хозяев и последуем за теми странными существами, которых Пурга и Ханоней отогнали от Книги мудрости, за Вэрсой и Бабой-Ёмой.
	
	Долго добирались они до Великой северной равнины, а теперь за несколько мгновений очутились далеко-далеко от неё, в той самой лесной глуши, из которой и вышли.
	
	Вэрсу Хановей забросил на густую вершину вековой сосны, откуда, запутавшись в ветках, горбун, несмотря на богатырскую силу и ловкость, полдня не мог спуститься на землю. А Баба-Ёма шлепнулась в болото и едва в нем не утонула. После долгих усилий, измазанные и исцарапанные, они добрались до небольшой поляны и немного передохнули.
	
	Оба уже успели забыть о недавнем споре из-за Книги мудрости и теперь думали лишь о том, как бы поскорее утолить голод.
	
	Вэрса, еще когда ураган крутил его над лесом, заметил поблизости коров, а Ёма вздумала наведаться в одинокую избу, затерянную в самой гуще пармы, так называют окружающее их зелёное царство коренные жители печорского севера, люди коми.
	
	В поисках добычи Баба-Ёма и Вэрса отправились в разные стороны, а встретиться условились на этой же поляне.
	
	Горбун пошёл к коровьему стаду, Баба — к тому домику, который видела с высоты птичьего полёта, когда Хановей нёс её над пармой. Много воспоминаний связывало Ёму с этим жилищем.
	
	Баба-Ёма была злой колдуньей, но не всегда она выглядела такой отвратительной старухой. Была и молодой и довольно красивой.
	
	Много всяких трав, полезных и ядовитых, знала Ёма. Умела варить из них душистые зелья, приготовлять всякие настои: умела и коров, и оленей лечить, умела и порчу насылать. Рассердится на кого-нибудь, вздумает ему навредить и как-то по-особенному поглядит своими злыми глазами на его дом. Смотришь: пройдет немного дней — дом сгорит. Глянет так же на соседскую корову — та молоко перестанет давать. Дети молока просят, плачут, а злюка только усмехается про себя, вот, мол, что я наделала. От одного её взгляда молоко сразу скисало, становилось невкусным, даже горьким.
	
	Никто не знал, из какой деревни пришла Ёма в эти края. Здесь познакомилась она с добрым охотником Иваном. Сумела хитрая Баба приколдовать его и поселилась е ним вместе в одинокой избушке. Но недолго царили здесь мир и согласие. Не могла Ёма жить спокойно. Часто покидала она жилище, чтобы наведаться в какую-нибудь деревню, навредить её жителям; иногда надолго углублялась в густую парму, вытаскивала добычу из чужих силков и канканов. В парме познакомилась она с Вэрсой, лешим.
	
	Злые люди рано стареют. Нехорошие мысли наложили отпечаток на лицо колдуньи. Нос у неё стал таким длинным, что нависал над верхней губой. Люди, заприметив издали колдунью, торопились обойти её стороной. Называть эту старуху стали Бабой-Ёмой и даже одним её именем пугали маленьких детей.
	
	Капризничает малыш, хнычет, спать не хочет, а то и встать утром ленится, мать его пугает:
	
	— Вовремя не встанешь или не ляжешь, или плакать не перестанешь — Бабу-Ёму позову. С большим мешком она придет, тебя заберет.
	
	Услышит малыш про колдунью и быстро успокаивается, слезы рукавом утирает, по сторонам оглядывается, спешит родительское распоряжение выполнить: боится Ёмы.
	
	Все удивлялись: как хороший человек, дед Иван, мог её терпеть? А он все сносил, старался образумить, чтобы поменьше зла она причиняла и уменье своё обращала на пользу людям, а не во вред.
	
	Но усилия пропадали даром. Баба даже грамоте учиться не захотела, а при её способностях она могла бы ей очень пригодиться. Колдунья веник в руки не брала, а наоборот, призывала в избу всяких жуков, тараканов и прочую нечисть. А иногда такую вонищу разведет колдовским варевом, что дышать становится нечем, стены и потолок покрываются жирными пятнами, а вокруг избы деревья начинают сохнуть.
	
	Кто знает, на сколько бы еще хватило терпения деда Ивана, если б в один зимний день, обходя силки, не услышал он детский плач, больше похожий на стон. Обернулся Иван и увидел на ёлке, возле занесенной снегом таёжной тропинки, маленькую девочку. Верно, в пути застигла её страшная беда: загрызли волки её родителей, а она, бедная, в страхе как-то вскарабкалась на дерево и, замерзая., сидела на нём.
	
	Бросился охотник к дереву, снял окоченевшую малютку, растёр снегом, закутал в свою одежду и быстро понёс домой.
	
	Думал старик, обрадует Бабу: детей-то у них самих не было. Думал, может, при виде малютки смягчится, сердце Ёмы, и станут они вдвоём растить сироту.
	
	Но чуть увидела Ёма девочку, затряслась от бешенства и заорала:
	
	— Что ты мне принёс?! Сейчас же отнеси её опять в лес! На что мне она?!
	
	Тут уж и дед Иван не выдержал: стукнул широкой ладонью по столу так, что изба ходуном заходила, а стол разлетелся в щепки. Ёма оробела, замолчала.
	
	Назвали сироту Марпидой. Тихая она была, не капризная. Но Баба исподтишка всячески старалась извести её. Уйдет дед на охоту, Баба по три дня сироте есть не дает, да еще побьёт ни за что ни про что. А как вернется дед, старуха ворчит, всякие нелепости наговаривает на малышку. Только дед не верил старой: знал её дурные привычки. А к девочке, послушной и ласковой, привязался старый охотник всем сердцем.
	
	Но заметил он, что Марпида худеет, бледнеет, чахнет: Ёма тайком её горьким зельем поила. Догадался об этом дед Иван, прикрикнул на Ёму, пристыдил её. А та взъярилась, схватила большой ковш, черпнула ядовитое зелье, которое как раз тогда варила, да и плеснула в лицо старому охотнику. Страшно закричал он ослепленный, так закричал, что весь лес откликнулся на тысячи голосов и услышали другие охотники, что случайно оказались поблизости, и отозвались. Испугалась Баба-Ёма, выскочила из избы и помчалась прочь в дебри таёжные, к Вэрсе. С тех пор долго её никто не видел.
	
	А дед Иван ослеп. Тяжело пришлось ему и маленькой Марпиде. Охотники вскоре ушли. Припасы кончались. Но тут, к счастью, выглянуло весеннее солнышко, потемнел и начал таять снег.
	
	Марпида оказалась на редкость умной и расторопной. Быстро научилась она силки на зверей и птиц ставить, рыбу ловить, сети чинить, пищу готовить, убирать в избе. Стало в ней чисто и уютно. Конечно, несмотря на все старания, девочка не могла сделать того, что делал дед, когда был зрячим, а потому они всё-таки жили бедно. Правда, понемногу он свыкся со слепотой, плел сети, выполнял кое-какую работу по дому, а главное, рассказывал Марпиде, что, где и как делать нужно. Многое знал и умел Иван, и многому научилась у него внимательная девочка. Подрастала она, с каждым днем умнела, хорошела и называла его отцом, батюшкой.
	
	В ту пору появился в тех местах знаменитый охотник Пера. Славился он силой и ловкостью, смекалкой и добротой. Как-то набрел он в лесу на одинокую избушку слепого, увидел юную Марпиду и стал им помогать.
	
	Полюбились молодому охотнику дед Иван и приветливая девушка. Нередко долгими зимними вечерами, когда за стеной гуляла Пурга, заходил в гости Пера посмотреть диковинные узоры, вышиваемые Марпидой, поглядеть, как она вяжет; слушал вместе с нею рассказы старика. От него узнал Пера о Книге мудрости и Камне жизни.
	
	Полагал старик, что чудесный Камень затерян где-то в снежных краях, и если люди найдут его, он принесет им счастье. А ведь Пера мечтал об этом. После рассказов деда Ивана молодой охотник уходил в парму, вслушивался в голоса птиц и зверей: не принесет ли кто из них весть о волшебном Камне. Но ничего не удалось выведать Пере. Отправился он по следам Ханонея. Полгода кружил по лесам и болотам, по рекам и тундре и опять-таки вернулся ни с чем. Не так-то легко было найти Камень жизни. Вы помните: Хановей и Пурга вдавили его глубоко в землю, засыпали ею, завалили снегом и льдом. Да и могла ли такая чудесная вещь лежать где-нибудь на виду?
	
	А Баба-Ёма, которая научилась не хуже охотников языку зверей и птиц, как-то услышала от болтливой сороки о Книге мудрости.
	
	— В ней написано, — верещала птица. — как найти волшебный Камень. Он и жжет и греет, и сжигает и оживляет, и лечит, и много чудес может сделать. А как им пользоваться, сказано в чудесной Книге, я сама её видела у Белой Совы. — затараторила лесная сплетница.
	
	Запомнились сорочьи слова Бабе-Еме. Поняла она, каким могучим станет обладатель волшебного Камня. Поняла она и то, что без Книги мудрости его не добыть, не узнать, где он спрятан. Решила Ема отыскать Книгу и завладеть Камнем.
	
	— Ох и много зла я смогу тогда сделать, — мечтала она, — разбогатею — и все будет мне подвластно — и реки, и леса, и люди, и звери. Всеми буду править я! Все будут мне служить и прислуживать.
	
	Вэрса тоже услышал о Камне жизни и Книге мудрости. Ему также захотелось добыть их, чтобы безнаказанно хозяйничать в лесном царстве и вредить добрым людям.
	
	Не сговариваясь, Баба-Ёма и Вэрса отправились разными путями на поиски Камня жизни к Великой северной равнине. Там мы с ними и познакомились, когда они попытались выманить Книгу мудрости у Белой Совы.
	
	\
	{%
		\centering
		\subsection*{\textcolor{red}{6. ПРИЗЫВ О ПОМОЩИ}}
	}
	
	\lettrine[findent=0pt]{\textbf{\textcolor{red}{К}}}{} огда Пера после безуспешных поисков пришел в избушку, Марпида вышивала полотенце. Заметила она, что Пера любуется её работой, и промолвила:
	
	— Тебе нравится — тебе и достанется. Закончу узор и дам тебе на память.
	
	Очень тронуло охотника внимание доброй девушки.
	
	Вдруг слепой приложил палец к губам. Все замолчали. В тишине слышалось только потрескивание горящих дров.
	
	— Не могу понять, что сегодня творится со мной? — тихо вымолвил дед Иван. — Почудилось, будто Она опять здесь и на меня оттуда, — он кивнул на угол за печкой, — с ненавистью смотрят её волчьи глаза.
	
	Марпида и Пера сразу догадались, что дед вспомнил о Ёме. Девушка побледнела, быстро подошла к печке, заглянула за неё:
	
	— Нет, батюшка, — прошептала Марпида, — никого здесь нет, да и не вернется Она сюда. И ведь с нами Пера...
	
	Девушка с надеждой посмотрела на охотника, тот слегка наклонил голову и улыбнулся.
	
	— Э-эх, хуже всего слепота, — протянул старик. — Сила еще есть, есть сноровка, уменье. А к чему они, когда не видишь, куда приложить их? Ничего не видишь...
	
	Девушка нежно прильнула головой к его плечу.
	
	Внезапно за стеной раздался собачий лай. Дверь отворилась. Худой мужичок в потрепанной одежде, едва переступив порог, шагнул к Пере, повалился ему в ноги и заголосил:
	
	— Выручи! Не дай погибнуть! Не дай помереть с голоду беспомощным детям! Только ты один можешь нас спасти!
	
	С мужичком вбежала большая лохматая собака, словно присоединяясь к мольбе хозяина, легла у ног Перы, и глядела на него ясными умными глазами.
	
	Пера поднял вошедшего и усадил на лавку. Марпида поставила перед ним еду и питье, дала поесть его лохматому спутнику. Мужичок понемногу пришел в себя и рассказал о том, что привело его сюда.
	
	Несколько дней назад возле его деревни появился страшный леший Вэрса и угнал всех коров, все стадо. Верная собака, которая предупредила о появлении вора, погибла от удара его тяжелой дубины. Немногочисленные жители деревни понимали, что сами не смогут одолеть могучего злодея. Они решили позвать на помощь Перу, и отправились во все стороны на розыски богатыря.
	
	— Кто еще справится с Вэрсой? На тебя вся надежда. Спаси нас и детей наших! — закончил грустный рассказ крестьянин.
	
	Уронил он голову на руки и заплакал. Дед и Марпида стали его успокаивать, а Пера задумался.
	
	Не боялся он лешего, никого не боялся, отдал бы свою жизнь, чтобы спасти других; всей душой хотел помочь беднякам. Но для этого нужно было найти Вэрсу. А легко ли увидеть лешего в его царстве? Он на дерево залезет — и не различишь его среди густых ветвей. Он сам в кует или дерево превратится — и ты мимо пройдешь, а он на тебя сзади набросится. Как же с ним встретиться?..
	
	Пера задумчиво потрепал по шее собаку. Та в ответ на ласку завиляла хвостом и доверчиво положила голову на колени богатыря.
	
	— Серко поможет тебе, — сказал крестьянин. — Он всюду поспеет и дорогу отыщет. Где мне, старому, не пройти, он пройдет. Возьми его себе на помощь.
	
	Крестьянин погладил собаку и что-то пошептал ей на ухо, указывая на Перу. Пес замотал хвостом.
	
	— Хотя бы коров отбить и вернуть беднякам, — подумал Пера; решительно поднялся, взял богатырскую палицу, лук и колчан со стрелами, подошел к крестьянину и попросил провести его к тому месту, где паслись коровы, когда нагрянул леший.
	
	— Оттуда, — решил Пера, — я пойду с собакой по его следам. `
	
	Лицо бедняка посветлело. Несмотря на усталость, он согласился сразу же пуститься в обратный путь. Марпида поспешно приготовила им на дорогу еду и вместе с дедом вышла проводить гостей.
	
	Сперва старик, держа под руку Марпиду, двигался довольно быстро, благо тропинка была знакомая и место ровное. Но через некоторое время, чтобы не задерживать Перу и его спутника, девушка и слепой решили вернуться и, не спеша, направились домой.
	
	А так как они идут медленно, мы задолго до их прихода успеем еще раз заглянуть в избушку, где познакомились с этими добрыми людьми.
	
	\
	{%
		\centering
		\subsection*{\textcolor{red}{7. БАБА-ЁМА ХОЗЯЙНИЧАЕТ}}
	}
	
	\lettrine[findent=0pt]{\textbf{\textcolor{red}{С}}}{} тарик не ошибся: за ним следили глаза Ёмы. Она не посмела приблизиться к двери, так как над нею висели большие щучьи зубы. А все древние коми знали, что это отпугивает всякую нечистую силу. Вэрса и Баба-Ёма это знали тоже и потому не решались заходить в те двери, над которыми висела раскрытая щучья пасть. Впрочем, Ёма не рискнула приблизиться к входу еще и потому, что её чуткое ухо уловило звуки незнакомого мужского голоса, доносившегося из избы. На всякий случай осторожная старуха подошла к дому с другой стороны, вынула из большой берестяной заплечной торбы, пестеря, как её называют коми, несколько длинных сосновых иголок, воткнула их в мох, чуть выступавший из-под одного бревна в стене, и шепнула колдовское слово. Бревно бесшумно отделилось от нижнего, на котором лежало, и чуточку приподнялось. В стене за печкой образовалась узенькая щёлочка. Ёма заглянула в неё, увидела Марпиду, Ивана, широкоплечего Перу и сразу догадалась, что это и есть тот знаменитый богатырь-охотник, о котором говорят в народе коми. Понимая, что в таком обществе ей делать нечего, Ёма бросила ненавидящий взгляд на слепого, и как раз этот взгляд он почувствовал. Но когда Марпида подошла к печке, послушные Ёме брёвна вновь неслышно сомкнулись, а сама колдунья, бормоча под нос разные ругательства, которые повторять неприлично, направилась в глубь леса. Однако не успела она пройти ста шагов, как услышала собачий лай.
	
	Старуха остановилась.
	
	Собака не обратила на неё внимания и пробежала мимо. Вслед за собакой появился запыхавшийся мужичок и вместе со своим хвостатым другом вошел в дом.
	
	— Это еще что за собрание?! — хмыкнула Ёма и хотела уйти подальше, но любопытство заставило её задержаться и последить за жилищем.
	
	Ждать пришлось недолго. Когда Пера и его спутники вышли из дверей, колдунья смекнула, что богатырь опять направляется в какой-то далёкий путь. Она поняла, что в избе никого не осталось, и едва голоса замолкли в отдалении, поспешно заковыляла к задней стенке дома. Там стала нашёптывать что-то и втыкать сосновые иглы между брёвнами. Подчиняясь колдовству, они начали раздвигаться. В стене образовался такой большой проём, что злодейка через него без труда проникла внутрь.
	
	— Фф-фу-у-уу — тяжело выдохнула она, когда очутилась посреди избы и огляделась.
	
	Всё здесь было ей не по нраву. Блестел тщательно вымытый и выскобленный пол. Лоснясь начищенными боками, на полочках стояли рядком глиняные горшки, деревянная и берестяная посуда.
	
	Ёма тоскливо поглядела на печку и стены, где не увидела даже ни одного таракана.
	
	— Всех вывела, — со злобой подумала старуха про Марпиду. — Всю живность истребила.
	
	Колдунья торопливо достала из заплечной торбы несколько больших мешков и принялась набивать их всем, что попадалось под руку. Желая не просто навредить добрым людям, а вообще сжить их со света, она заглянула в погреб, в чулан и сарай возле дома. Там хранились съестные припасы на долгую зиму. Грабительница не оставляла ни крошки, ни косточки, все складывала в мешки.
	
	Посуда не захотела перейти в руки Ёмы. Все глиняные горшки при её приближении разом грохнулись с полок и печки на пол и разбились на тысячи черенков. Один, самый пузатый горшок попытался свалиться прямо на злодейку, но неудачно перевернулся и уселся ей на голову, как шапка. Старуха его не разбила, а так и оставила на голове вместо горшка, разбитого при ссоре с Вэрсой из-за Книги мудрости. Все охотничьи и рыболовные снасти, стрелы, длинное копье с серебряным наконечником — всё захватила негодница. Что не могла взять — поломала. Особенно обрадовали её кремень и огниво.
	
	— Без них огня не добыть. Пусть замерзнут, — злорадно хмыкнула Баба-Ёма и затолкала находку в мешки. Заполнив их доверху, она самодовольно подбоченилась, любуясь опустошением, и прислушалась. Ничто не нарушало тишины.
	
	— А что если сжечь избу?! — мелькнуло в голове злодейки. — Но огонь и дым видны далеко и Пера и жильцы могут поторопиться с возвращением.
	
	— Нет. — нахмурилась Ёма, — пусть голодают в холодной избе. И она вылила на тлевший в печке огонь всю воду из берестяных вёдер. Затем Ёма с трудом протиснулась в дверь с награбленным добром и бросила торжествующий взгляд на щучью пасть, которая могла помешать только войти в дом, но не выйти из него.
	
	Быстро темнело. Того и гляди, вернутся хозяева, — подумала воровка. Уже за порогом она накрепко связала друг с другом узлы, столкнула в реку все вёдра, кадки, все заготовленные на зиму дрова и подошла к высокой берёзе у дома.
	
	Что нашёптывала дереву колдунья, неизвестно. Но, подчиняясь заклинаниям, оно низко наклонилось. Ёма привязала к крепкому суку на вершине ствола огромный ком награбленного, отошла на почтительное расстояние, что-то пробормотала и протянула руки в том направлении, откуда пришла.
	
	Внезапно дерево резко распрямилось, мешки и узлы сорвались с его вершины и понеслись по воздуху туда, куда указывала колдунья. Она довольно крякнула, трижды плюнула через правое плечо на опустошенную избу и с неожиданным проворством быстро зашагала через лес к той поляне, где условилась встретиться с Вэрсой.
	
		\
	{%
		\centering
		\subsection*{\textcolor{red}{8. ВОЗВРАЩЕНИЕ}}
	}
	
\lettrine[findent=0pt]{\textbf{\textcolor{red}{Г}}}{} устые тучи опускались на верхушки деревьев. Сырой осенний ветер дул в лица старика и Марпиды. Оба устали и проголодались. Уже стемнело, когда они подошли к дому.
	
	— Хорошо бы сейчас погреться и поужинать... горяченького... — заметил слепой.
	
	Марпида не ответила.
	
	— Странно, — подумала она, сразу приметив открытую настежь дверь.
	
	Едва они переступили порог, как пахнуло сыростью: Ёма залила огонь в печке и расплескала воду по полу.
	
	Дед Иван пожал плечами в недоумении, затем попытался осторожно приблизиться к печке. Под ногами затрещали глиняные черепки от разбитой посуды.
	
	— Кто тут хозяйничал?.. — дед остановился, вытянул вперед руки, сделал неуверенно шаг, другой, коснулся ладонями холодной печки и тихо сказал; — Надо поскорее развести огонь.
	
	Слепой шарил руками по печке, но кремня, огнива, трута и бересты на обычном месте не находил.
	
	Глаза Марпиды уже привыкли к темноте. Девушка спустилась в погреб, обшарила чулан, вышла во дворик — и поняла все: их обворовали; даже дрова и те исчезли. У Марпиды подкосились ноги. Она бессильно прислонилась к высокой берёзе. Резкий порыв ветра ударил в лицо девушке.
	
	— Марпида, где ты?! — услышала она голос старика и подняла глаза. Над её головой колыхалось что-то белое. Это было полотенце, то самое, которое она с такой любовью вышивала для Перы. Марпида осторожно сняла е ветки эту единственную, случайно уцелевшую вещь и вошла в дом.
	
	Нечем было добыть огонь, нечего было приготовить даже на самый скромный ужин.
	
	— Батюшка, милый, что мы будем делать? — сквозь слезы сказала девушка, обнимая старика. — Какой-то злодей ограбил нас дочиста, ни сушеного грибка, ни зёрнышка, ни ягодки не оставил!
	
	Слепой гладил мягкие волосы Марпиды. Что он мог сказать? Чем мог утешить? Пера ушел надолго. Надеяться было не на кого и не на что.
	
	За разбитым окном всё яростнее гудел холодный ветер, и снег серебряным покрывалом ложился на землю. Начиналась зима.
	
			\
	{%
		\centering
		\subsection*{\textcolor{red}{9. ЖИРНЫЕ МЫШИ И КНИГА МУДРОСТИ}}
	}
	
	
	\lettrine[findent=0pt]{\textbf{\textcolor{red}{В}}}{} эрса загнал коров в глухое урочище, напился вдоволь молока и бросил стадо, а сам побежал к месту свидания с Ёмой. Вскоре он очутился на берегу довольно широкой реки. Перепрыгнуть через неё леший не отважился. Переходить вброд он тоже не хотел, потому что не ладил с повелителем рек и озер Ва-Кулем. Возможно, из-за этого леший никогда не умывался, его тело покрылось густым слоем грязи, спина и грудь заросли мохом.
	
	Вэрса раздумывал, как переправиться через реку, пока не заметил в зарослях ивняка, недалеко от себя, лодку-долблёнку. На дне её лежали сети и весло. Обрадованный Вэрса вскочил в неё, оттолкнулся от берега и переправился на другую сторону. Неизвестного хозяина лодки он отблагодарил тем, что выкинул за борт сети, забросил далеко в лес весло, а ни в чем не повинную лодку перевернул и вытолкнул на волю течения, в самую стремнину. Эта проделка привела лешего в такой восторг, что, добравшись до места встречи с Бабой-Ёмой, он в ожидании подруги начал кувыркаться и, наконец, опустился на четвереньки и захрюкал от удовольствия так громко, что испуганные вороны с тревожным карканьем вылетели из гнезд и полетели прочь…
	
	Неизвестно, сколько еще времени продолжал бы он подобные занятия, если б сильный удар чем-то большим и мягким по нижней части спины не заставил безобразника уткнуться носом в землю.
	
	Рядом шлёпнулось еще что-то и еще... Леший скосил единственный глаз и увидел несколько больших узлов и мешков. Они явно свалились откуда-то сверху. Вэрса прыгнул под ближайшее густое дерево и с опаской посмотрел на небо. Больше ничего не падало. Однако он оробел и решил дождаться наступления темноты, чтобы ближе познакомиться с таинственными мешками. Он присел под сосной и не спускал с них глаза: а что если снова улетят?..
	
	Вдруг ветки кустов раздвинулись. На поляну вышла Баба-Ёма. На её голове красовался новый глиняный горшок, на шее, как ожерелья, болтались связки сушеных грибов, за поясом торчали топор, пара ножей, несколько деревянных ложек. Через плечо колдуньи были небрежно перекинуты шкурки лисиц вперемежку с вяленой рыбой, а на ногах пестрели мягкие сапожки из оленьих шкур.
	
	Опиралась Ёма на копьё с блестящим серебряным наконечником. Она удовлетворённо глянула на мешки и узлы: все в целости. Только теперь леший догадался, откуда они взялись, и подскочил к старухе.
	
	— Ну? Что? Откуда? У кого? — посыпались отрывистые вопросы.
	
	— У слепого, — с недоброй усмешкой ответила Ёма и стала развязывать узлы.
	
	Глаз лешего разгорелся: кроме охотничьих и рыболовных снастей, мехов, одежды и украшений, к которым он был равнодушен, прожорливый Вэрса увидел множество съестных припасов. Дрожа от жадности, он сразу набросился на пищу, заталкивая в огромный рот оленьи ножки, солёную рыбу с костями, сушеные грибы, сладкие ягоды... Только желудок лешего мог переварить подобную смесь.
	
	Напрасно подруга пыталась удержать ненасытного приятеля. Наконец, он наелся и, тяжело дыша, сел на пенёк, поглаживая руками раздувшийся живот.
	
	Старуха попросила помочь ей унести и припрятать награбленное. Но леший отказался. Зная, что спорить с ним бесполезно, Ёма в сердцах плюнула и сама принялась укладывать добычу.
	
	Вдруг Вэрса легонько толкнул подругу в бок и указал грязным пальцем куда-то вдаль, поверх сосен. Там, над их верхушками, появилось маленькое пятнышко. Приближаясь, оно всё увеличивалось.
	
	Ёма подняла голову:
	
	— Она, — прошептала колдунья.
	
	— Тяжело летит, — заметил Вэрса. — Устала.
	
	— Заманить бы. — прогундосила Ёма. — Вдруг при ней…
	
	Она знаком приказала хранить молчание, приложила ладони рупором ко рту и, подражая голосу совы, закричала:
	
	— У-гу-у! У-гу-у! Лети сюда, Белая Сова!.. Не сердись на нас! У нас есть вкусные жирные мыши. Милости просим в гости, Белая Сова!..
	
	Действительно, над лесом летела наша старая знакомая. Получив приказ доставить Книгу мудрости на Тэв-Поз-Из, она отлетела подальше от Пурги и Хановея, а потом, зная, что они еще надолго задержатся в тундре, наведалась к своим пернатым родственникам. В их обществе Сова посетовала на своих неотесанных хозяев, немного отдохнула, нежно поцеловала чужих птенцов, простилась и полетела дальше. В гостях она почти не ела, только попробовала пару вкусных ягод, заготовленных на зиму.
	
	В ожидании морозов лее обнажился, и птица с небольшой высоты могла его хорошо просмотреть. Увы, никакой добычи для себя, даже захудалого зайчишки, Сова не заметила. И тут она услышала крик Ёмы. Осторожная птица была невысокого мнения о колдунье и её спутнике, которые вздумали клянчить у неё Книгу мудрости. Но... известие о жирных мышах, её любимом блюде, заставило Сову позабыть о недавнем приключении.
	
	— А кто их знает? — подумала она. — Что если от чистого сердца хотят угостить, благо сами мышей не едят. Может, их совесть мучает за то, что так грубо пытались выманить у меня Книгу?.. Да и что они могут мне сделать, если не умеют ни читать, ни летать?.. А, может, лучше поостеречься?..
	
	Но голос благоразумия заглушило урчание пустого желудка и, ради жирных мышей, хранительница Книги мудрости опустилась на поляну.
	
	Она поклонилась Ёме и Вэрсе и вопросительно посмотрела на них: где мыши? Колдунья сразу поняла взгляд птицы и засуетилась:
	
	— Сейчас, сейчас!
	
	При этом хитрая старуха поспешно сняла с себя одно ожерелье из сушеных грибов и положила на землю. Сова была не прочь их попробовать, но еще больше её интересовали мыши.
	
	— Что там у тебя под крылом, Книга мудрости? — как бы невзначай спросила Ёма, снимая с нитки гриб и любезно поднося его к самому клюву птицы.
	
	— У-гу, — утвердительно кивнула Сова и слегка отодвинулась от чрезмерно внимательной Бабы-Ёмы.
	
	— Ах, — старуха скроила умильную рожу, — хоть бы посмотреть поближе, полюбоваться ею.
	
	— Нельзя, — сухо возразила Сова. — А где мыши?
	
	— Есть! — воскликнула Ёма. — Есть и будут мыши! Кстати, сколько мышей стоит такая книга? — полюбопытствовала она, стараясь заглянуть под крыло.
	
	— Книга мудрости стоит дороже мышей, — многозначительно заметила Сова.
	
	— Дай-ка мне подержать её!
	
	С этими словами нетерпеливый леший неожиданно ухватился за уголок переплета, торчавший из-под крыла. Одновременно Ёма, как пружина, резко выпрямилась, подалась вперед и навалилась на птицу. Та попыталась поднять крылья, чтобы взлететь, но за ремешок, на котором была Книга, успела ухватиться колдунья. Сова ударила её своим страшным клювом. Злодейка пошатнулась и упала. Но тут же вскочила на ноги и бросилась к Сове с пустым мешком, стараясь накинуть его на голову птице. Та увернулась и снова попыталась взлететь. Тут Вэрса вцепился в неё. Сова клюнула и его. Он дико вскрикнул и, забыв о Книге, первой попавшейся под руку веткой ударил птицу по крылу в то самое мгновение, когда она, собрав все силы, оторвалась от нападающих и взлетела.
	
	— Далеко не уйдет: я ей крыло перебил. — злобно прохрипел вслед Вэрса, держась за рассеченное клювом плечо. Вместе с Ёмой он попытался преследовать Хранительницу, но вскоре махнул рукой.
	
	Сова скрылась за деревьями.
	
	Преследователи вернулись.
	
	— Все ты виноват: поторопился. — буркнула Ёма.
	
	Вэрса не возражал. Они потащили мешки с награбленным в глубь леса и решили на другой день хорошенько осмотреть чащу в надежде найти Книгу мудрости и тяжелораненую Сову.
	
	На полянке, где за короткое время произошло столько бурных событий, воцарилась тишина.
	
	Кряхтя под тяжестью ноши, Баба-Ёма и Вэрса брели по лесу. Они уже порядочно прошли, когда Баба-Ёма вдруг остановилась и спросила спутника:
	
	— А копьё?
	
	Вэрса никак не мог вспомнить, где он его оставил. Разгорелся спор. Он закончился тем, что леший переложил весь груз на спину подруги и отправился в обратный путь разыскивать копьё, к уже знакомой нам поляне.
	
	
	
\end{document}


