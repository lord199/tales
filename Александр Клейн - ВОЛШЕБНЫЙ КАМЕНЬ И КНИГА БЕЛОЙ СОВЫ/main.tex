
\documentclass[oneside,final,14pt]{extreport}
%\usepackage[koi8-r]{inputenc}
\usepackage[russianb]{babel}
\usepackage{vmargin}
\setpapersize{A4}
\usepackage[T2A]{fontenc}
\usepackage[utf8x]{inputenc}  % more recent versions (at least>=2004-17-10)
%\usepackage[russian]{babel}
\setmarginsrb{2cm}{1.5cm}{1cm}{1.5cm}{0pt}{0mm}{0pt}{13mm}
\usepackage{indentfirst}
\usepackage{nicefrac} % For comparison
%\usepackage{xfrac}    % Works better with other fonts
%\usepackage[unicode, pdftex]{hyperref}
\usepackage{lettrine}
\usepackage[usenames]{color}
\usepackage{colortbl}
\usepackage[pagestyles]{titlesec}
\usepackage[colorlinks=true,linkcolor=black,urlcolor=black,bookmarksopen=true]{hyperref}


% Настройка вертикальных и горизонтальных отступов
\titlespacing{\chapter}{0pt}{5pt}{5pt}
\titlespacing{\section}{\parindent}{4mm}{4mm}
\titlespacing{\subsection}{\parindent}{3mm}{3mm}



\sloppy
\begin{document}
	
	\section*{Описание}
	
	{\bf Название:} Волшебный Камень и Книга Белой Совы
	
	{\bf Автор:} Александр Соломонович Клейн
	
	{\bf Издательство:} Сыктывкар: Коми книжное издательство, 1986
	
	{\bf Художник:} Г. С. Тонков
	
	{\bf Редактор:} А. В. Некрасов
	
	{\bf Художественный редактор:} В. Б. Осипов
	
	{\bf Технический редактор:} А. Н. Вишнева
	
	{\bf Корректоры:} А. А. Надуткина, Т. И. Попова
	
	{\bf Аннотация:} Эта повесть-сказка, написанная на основе древних коми поверий и преданий, посвящена удивительным приключениям добрых и смелых людей, их борьбе против злых сил за право владеть волшебным камнем и Книгой мудрости.
	\thispagestyle{empty} % выключаем отображение номера для этой страницы
	\newpage
	
	\begin{titlepage}
		
		\begin{center}
			%\vfill
			
			%\vfill
			\topskip0pt
			\vspace*{\fill}
			
			
			{\large\bf Александр Клейн\\}
			\ \\
			\ \\
			{\Huge\bf ВОЛШЕБНЫЙ КАМЕНЬ И КНИГА БЕЛОЙ СОВЫ\\}
			\ \\
			\ \\
			\ \\
			Повесть-сказка
			\ \\
			\ \\
			По мотивам коми фольклора
			\vspace*{\fill}    
			
			\vfill
			
			Сыктывкар
			
			Коми книжное издательство 1986
		\end{center}
		
	\end{titlepage}
	
	\thispagestyle{empty} % выключаем отображение номера для этой страницы
	
	\newpage
	
	\begin{flushleft}
		\begin{verse}
			\qquad \qquad Читатель-друг, ступай со мной\\
			\qquad \qquad Туда, где лес стоит стеной\\
			\qquad \qquad И птицы людям говорят\\
			\qquad \qquad Про вьюжный край, где спрятан клад,\\
			\qquad \qquad Где серебром блестят равнины,\\
			\qquad \qquad А в горных облачных вершинах\\
			\qquad \qquad Гнездятся ветры.\\
			\qquad \qquad\qquad\qquad\qquad\qquad Труден путь.\\
			\qquad \qquad Но не робей и рядом будь.\\
			\qquad \qquad Знай: одному — и мне беда.\\
			\qquad \qquad С тобой — согласен хоть куда;\\
			\qquad \qquad Один — тяжел я на подъём.\\
			\qquad \qquad А вместе — в сказку мы войдём!\\
			\qquad \qquad 
		\end{verse}
	\end{flushleft}

\setcounter{secnumdepth}{0}  
	
	\section[1. Как исчезло зелёное царство]{\center \textcolor{red}{1. КАК ИСЧЕЗЛО ЗЕЛЁНОЕ ЦАРСТВО}}
	
	\lettrine[findent=0pt]{\textbf{\textcolor{red}{К}}}{}аждые сто лет в Книге мудрости прибавлялся только один лист. Никто не мог объяснить, как это происходило, но с каждым столетием Книга становилась чуть толще. Она была написана не совсем обычно, говорят, даже стихами. В ней было много полезных советов. Но никто из людей её никогда не видел. А о её существовании знали давно. Вероятно, старые охотники, хорошо понимающие язык птиц, услышали о чудесной Книге от них. Птицы не умеют хранить тайны, особенно сороки и воробьи. Они крошки не клюнут без того, чтобы не пискнуть, не сказать что-нибудь на своём языке, не посплетничать.
	
	А птицы, безусловно, должны были знать о Книге мудрости, потому что её хранительницей была тоже птица — Белая полярная Сова. Вы её видели когда-нибудь? Огромная, круглоголовая, с большими жёлто-огненными глазами и загнутым книзу, наподобие крючка, острым клювом; с могучими широкими крыльями, покрытыми маленькими коричневыми черточками и крапинками. У неё очень тонкий слух и отличное чутье. Неправду говорят, будто Белая Сова хорошо видит только ночью. Нет, она великолепно видит и днем. Ведь там, где она обитает, в бескрайних северных равнинах, половину года царит день, а ночи в это время вообще нет. Ночь там только зимой.
	
	Почётная должность хранительницы Книги мудрости передавалась совами из рода в род. Никто не помнит — когда, сколько тысяч лет тому назад пра-пра-пра-пра-пра-прабабушка одной Белой Совы, — а совы минут очень долго, — вложила в золотистый переплёт из непромокаемой нерпичьей шкуры первый лист заветной Книги. Никто не видел и того, кто сделал этот чудесный переплёт. Но... так уж случилось, что когда-то, давным-давно, бессмертные властители сурового и прекрасного севера, снежная буря Пурга и злой ветер Хановей, сами не умея читать, доверили белым совам хранение Книги мудрости.
	
	Раз в сто лет Хановей и Пурга интересовались тем, что в ней написано, и Сова читала им вслух о богатствах подвластной им земли, о людях, которые живут в этих холодных краях.
	
	После чтения Сова уносила Книгу в какое-нибудь потайное место, а Хановей и Пурга от безделия снова начинали кружить по просторам, заносить снегом стойбища оленеводов, морозить и сбивать с дороги одиноких путников. Наподобие исполинских чумов вздымались сугробы, оглушительно свистел ветер, снежные вихри закрывали небо. Тяжело приходилось тому, кого настигали Хановей и Пурга среди Великой северной равнины, тундры.
	
	Не всегда она была такой безжизненной. На месте болотных кочек, ложбин и голых холмов, покрытых чахлым кустарником, когда-то буйно зеленели травы, высились могучие деревья; на сказочно красивых цветах, как на волнах, пестрея крыльями, покачивались огромные бабочки. Прозрачный воздух был напоён душистой свежестью и, казалось, даже слегка вздрагивал от птичьих трелей.
	
	А посреди этого зелёного царства искрился и сиял Камень, несравненно прекраснее и дороже всех драгоценных камней — волшебный Камень жизни. Его щедрые лучи оживляли все вокруг, несли тепло и свет, радость и счастье. И под пение птиц мирно дремали чудо-леса в лучах доброго Камня.
	
	Очень давно это было... Так давно, что ученые люди до сих пор все считают и никак не могут точно сосчитать, сколько лет прошло с тех пор. Не могут они точно назвать не то что день, а даже месяц или год, когда впервые прилетели в этот край Хановей и его сестра Пурга. Они были такие же, как сейчас, лохматые, шумливые, неряшливые. Седая косматая Пурга и тогда была старухой. Она, говорят. сразу родилась старой. Да и братец ее Хановей был не моложе.
	
	Как и откуда они залетели? Неизвестно. Во всяком случае, не из жарких мест, потому что оба терпеть не могли, не выносили ласкового тепла и света. Но увидели они окружающую красоту, услышали пение птиц и невольно замерли. Уж на что был груб и неотёсан Хановей, и тот не выдержал: глубоко вздохнул, залюбовался и на его сосулистую бороду, дрожа, закапали слезинки умиления.
	
	Но Пурга быстро вывела брата из оцепенения.
	
	— Чего уставился, бородач? — прохрипела она. — С твоих усов скоро ручьи потекут и мокрое место от тебя останется, а ты рот разинул. Ведь этакая жара нас вот-вот погубит. Я задыхаюсь от запаха цветов и глохну от птичьего гомона. Пора положить конец безобразию.
	
	— Смотри!-— она указала на сверкающий Камень жизни. — Вот виновник всей этой кутерьмы. Его нужно убрать. Слышишь?! Только тогда мы сможем свободно разгуливать по земле. Ну-ка, принимайся за дело.
	
	С этими словами злая старуха дико завыла, затрясла косматой головой — и на заветный Камень, на зеленые деревья, на цветы повалил снег.
	
	Хановей не отстал от сестрицы: он засвистел по-разбойничьи и, откуда ни возьмись, на цветущую землю обрушился страшный ураган. Он ломал кусты и деревья, вдавливал их глубоко в землю, и она, еще недавно такая теплая и податливая, становилась твердой, как лёд, который, подчиняясь приказу Пурги и Хановея, лавинами наступал на зеленое царство.
	
	Все свои силы собрали Хановей и Пурга против Камня жизни. Долго его чудесные лучи не позволяли врагам приблизиться к нему, расплавляли снег и лед, пытались посылать живительное тепло окружающим лесам. Злые пришельцы отступили. Но ненадолго. Пурга исчезла лишь затем, чтобы подкрасться сзади, со стороны Большого холодного моря и окатить Камень жизни гигантскими волнами. А Хановей взобрался на высокие горы, дохнул во всю мощь морозным ветром, и волны, нахлынув на Камень, мгновенно застыли, сковали его толстым льдом и вдавили глубоко в землю.
	
	Мигнул и исчез последний луч... Загоготали от радости злые силы, заплясали над холмами, покрывшими волшебный Камень. От прекрасных лесов остались только редкие кусты да согнутые в три погибели карликовые берёзки.
	
	Так воцарились на севере Хановей и Пурга. Седые, смертельно бледные, они больше всего любили белый цвет, а потому щедро осыпали свои владения снегом. И под цвет снега, боясь грозных хозяев, стали рядиться и зайцы, и песцы, и куропатки, и другие звери и птицы, одним словом, почти все жители бескрайной северной равнины, тундры.
	
		\section[2. Условный знак]{\center \textcolor{red}{2. УСЛОВНЫЙ ЗНАК}}
	
	\lettrine[findent=0pt]{\textbf{\textcolor{red}{П}}}{}рошло много тысяч лет...
	
	\qquad	Чувствуя себя безраздельными хозяевами, Пурга и Хановей стали беспечнее, обленились. Хановей завел себе быстрейших оленей, и стремительно мчали они его, куда ему заблагорассудится. Порой он сажал в свои серебряные сани-нарты Пургу, и катались они вдвоем вволю.
	
	Но, хотя свободного времени хватало, учиться они ничему не захотели, даже грамоте. Зачем она им. когда у них есть Белая Сова? Та и читать и даже считать умеет.
	
	Как-то накануне окончания длинной зимней ночи Пурга заметила в нечёсаной бороде Хановея совиное перо с тремя темными точками. Это был условный знак.
	
	— Белая Сова вызывает нас. — сказала Пурга, осмотрев перо. — Значит, завтра исполнится ровно сто лет с того дня, как мы в последний раз заглядывали в Книгу мудрости. Пора опять заглянуть.
	
	— А что там может быть нового? — буркнул Ханоней. — Нас-то она не касается.
	
	— Как знать? — покачала головой Пурга. — Книга — такая вещь, от которой всегда можно ожидать неожиданное, даже нам. Отправимся к Сове. Посмотрим, что она делает.
	
		\section[3. Новые знакомые]{\center \textcolor{red}{3. НОВЫЕ ЗНАКОМЫЕ}}

	\lettrine[findent=0pt]{\textbf{\textcolor{red}{К}}}{}азалось, Белая Сова в то утра надела королевскую горностаевую мантию. Все пёрышки были приглажены, вычищены до блеска. По виду никто бы не подумал, что красавице уже за двести лет: она выглядела, по крайней мерь, в два раза моложе. Время от времени в её больших круглых глазах вспыхивали язычки жёлтого пламени: отблески лучей далекого солнца. Оно перед зимней спячкой хотело проститься с тундрой.
	
	Сова сидела на лохматой кочке. Позади начинались заросли кустарника. Из него робко выглядывали посиневшие от ранних морозов горбатые ивы и костлявые берёзки.
	
	Но вот над гребнем одного из холмов появилась узкая багрово-красная лента: краешек солнца. Оно медленно, с трудом приподнялось над снежной пустыней. В то же мгновение Сова почувствовала под левым крылом чуть заметный толчок: в Книге мудрости прибавился еще один лист.
	
	Сова бережно вынула большую теплую Книгу из-под крыла, положила перед собой на заранее расчищенное место, на другую высокую кочку, и раскрыла.
	
	Сейчас лучи солнца озарят свежую, еще чистую страницу, и на ней выступят буквы. Тогда можно будет узнать, что вошло в вечную Книгу за сто лет.
	
	Сова поглядела по сторонам и склонилась над страницей. Страница слегка потемнела. Как сквозь туман, на ней проступали очертания отдельных букв. А Пурги и Хановея все не было.
	
	— Плохо, когда хозяева такие невоспитанные и неаккуратные, а доверить великую тайну предков некому, — горестно подумала Сова, и вдруг услышала, как за её спиной сухо треснули мерзлые ветки кустарника.
	
	— Доверь мне тайну своих предков. — словно читая её мысли, просипел над ухом птицы простуженный голос.— ‘Только доверь — и я открою тебе двери моего вечнозеленого царства.
	
	Сова стремительно расправила крылья над Книгой мудрости и резко повернула круглую голову назад.
	
	Там стояло странное горбатое существо — то ли человек, то ли зверь, то ли дерево?.. Из-под огромной зелёной шапки-шишки торчали мохнатые уши; посреди лба одиноко блестел круглый глаз; могучие плечи переходили в длинные руки. Они сжимали толстую дубину. На кривых ногах, одна из которых напоминала копыто, виднелись бесчисленные бородавки.
	
	— Ну, чего смотришь? Отдай! Не пожалеешь, — настойчиво повторил незнакомец.
	
	— Отдай мне, — тут же прошипел другой голое, и из-за кустов вышла скрюченная длинноносая старуха. Одета она была в пестрые лохмотья, а на её голове кокетливо покачивался щербатый глиняный горшок.
	
	— Отдай мне, — вкрадчиво продолжала старуха и протянула жилистую руку к Книге. — И я до скончания дней твоих дам тебе угол и пищу в жаркой избе.
	
	Одноглазый резко выпрямился:
	
	— Прочь!
	
	Старуха отпрянула в сторону и споткнулась. Горшок упал с её головы и разбился.
	
	— Тьфу, Вэрса безмозглый! — выругалась она. — Да зачем тебе Книга мудрости? Ты же глуп, как самый последний черепок этого разбитого горшка. Отдай мне Книгу. — прогнусавила она, приближаясь сбоку к Сове.
	
	— Нет, мне! — зарычал Вэрса. — Уйди, Баба-Ёма! Уйди!
	
	Но грозно поднятая дубина вдруг вырвалась из крючковатых пальцев Вэрсы и отлетела далеко в сторону.
	
	— Во-он! Во-он! — внезапно завыло и засвистело вокруг. — Во-он отсюда! Мы — хозяева Великой северной равнины! Во-он!..
	
	Откуда ни возьмись, налетел страшный вихрь. Снежная вьюга в одно мгновение ослепила пришельцев. Бешеный ветер сбил их с ног, закрутил, приподнял и понес прочь в далекие темные леса.
	
	Пурга и Хановей подоспели к Сове. Она все еще лежала на Книге, прикрывая её всем телом. Она поднялась только тогда, когда хозяева прогнали Вэрсу и Бабу-Ёму.
	
		\section[4. Тревога]{\center \textcolor{red}{4. ТРЕВОГА}}
	
	\lettrine[findent=0pt]{\textbf{\textcolor{red}{Н}}}{}-н-н-у-у-у, — спросила Пурга, — что там у тебя?
	
	\qquad  	 — Вот, — Сова торжественно опустила правую ногу на открытую страницу. На ней сверкали позолотой свежие строчки. Каждая начиналась с крупной, очень красивой заглавной буквы.
	
	Хановей шумно вздохнул и посмотрел на сестру. Та равнодушно кивнула Сове:
	
	— Читай.
	
	Но первые же слова новой страницы встревожили хозяев севера.
	
	— Кто чудесным Камнем жизни,
	
	Страх отбросив, завладеет, — ровным голосом читала Сова.
	
	— Опять об этом негодном Камне! — возмущенно топнул ногой Хановей. — Да когда же?!.
	
	— Погоди, — перебила Пурга. — Слушай дальше.
	
	— Тот бесценный дар получит. — продолжала читать Сова.—
	
	И погубит злые силы,
	
	Что над тундрою царили.
	
	— Постой! — не выдержал Хановей. — Повтори.
	
	Сова исполнила желание властелина.
	
	— Что скажешь, братец? — тихо спросила Пурга.
	
	Хановей, сердито сопя, уткнулся ледяным носом в Книгу.
	
	— А-а-а?.. — вопросительно протянула Пурга. Хановей не ответил.
	
	— Давай-ка, подумаем, — прошамкала Пурга. — Как ни трудно, а думать иногда нужно. Помолчим и подумаем. Подумать стоит крепко...
	
	Все потупили головы и так просидели три дня и три ночи.
	
	Первый нарушил молчание Хановей.
	
	— Я придумал! — гаркнул он во все горло. — Надо уничтожить Книгу.
	
	Пурга так затряслась от беззвучного смеха, что с её седой головы во все стороны полетели снежные вихри.
	
	— Ох-хо-хо! — застонала она. — Ты же знаешь: этого сделать нельзя. Что попало в Книгу мудрости, то в огне не горит и в воде не тонет. Когда ты это поймешь?
	
	— А если разорвать Книгу?!
	
	При этих словах Сова с возмущением посмотрела на невежду, а Пурга печально заметила:
	
	— Что ж, разлетятся её страницы, и кому-нибудь попадет эта, — старуха кивнула на новый лист. — Люди найдут его и отправятся за Камнем жизни.
	
	— Но что же делать, что делать, что делать? — зачастил Хановей. — Мы же погибнем.
	
	— Пожалуй, следует поступить так, — отделяя слова, отчеканила Пурга. — Нужно подальше спрятать Книгу, чтобы никто не смог её найти. Белая Сова! — властно обратилась она к молчаливо следившей за своими хозяевами птице. — Отнеси Книгу мудрости каменное гнездо ветров, на непреступную вершину Тэв-Поз-Из!
	
	— На Тэв-Поз-Из! — как эхо, повторил Хановей. — На Тэв-Поз-Из!
	
	— Там, — продолжала злая старуха, — я прошью её страницы вечным холодом.
	
	—А я обрадовался Хановей, — скую их вечным льдом и не так-то просто будет найти Книгу и развернуть её листы. И. удивляясь собственной догадливости, бородач несколько раз подпрыгнул на месте.
	
	Чего смотришь? — вдруг накинулась Пурга на Сову. — Сказано: на Тэв-Поз-Из — и все тут. Лети!
	
	Сова решила про себя, что обязательно дочитает страницу в более спокойной обстановке, аккуратно закрыла Книгу, обвязала её мягким ремешком, перекинула через плечо и пустилась в путь.
	
		\section[5. Колдунья и слепой]{\center \textcolor{red}{5. КОЛДУНЬЯ И СЛЕПОЙ}}

	
	\lettrine[findent=0pt]{\textbf{\textcolor{red}{Т}}}{}еперь оставим на время Белую Сову и её хозяев и последуем за теми странными существами, которых Пурга и Ханоней отогнали от Книги мудрости, за Вэрсой и Бабой-Ёмой.
	
	Долго добирались они до Великой северной равнины, а теперь за несколько мгновений очутились далеко-далеко от неё, в той самой лесной глуши, из которой и вышли.
	
	Вэрсу Хановей забросил на густую вершину вековой сосны, откуда, запутавшись в ветках, горбун, несмотря на богатырскую силу и ловкость, полдня не мог спуститься на землю. А Баба-Ёма шлепнулась в болото и едва в нем не утонула. После долгих усилий, измазанные и исцарапанные, они добрались до небольшой поляны и немного передохнули.
	
	Оба уже успели забыть о недавнем споре из-за Книги мудрости и теперь думали лишь о том, как бы поскорее утолить голод.
	
	Вэрса, еще когда ураган крутил его над лесом, заметил поблизости коров, а Ёма вздумала наведаться в одинокую избу, затерянную в самой гуще пармы, так называют окружающее их зелёное царство коренные жители печорского севера, люди коми.
	
	В поисках добычи Баба-Ёма и Вэрса отправились в разные стороны, а встретиться условились на этой же поляне.
	
	Горбун пошёл к коровьему стаду, Баба — к тому домику, который видела с высоты птичьего полёта, когда Хановей нёс её над пармой. Много воспоминаний связывало Ёму с этим жилищем.
	
	Баба-Ёма была злой колдуньей, но не всегда она выглядела такой отвратительной старухой. Была и молодой и довольно красивой.
	
	Много всяких трав, полезных и ядовитых, знала Ёма. Умела варить из них душистые зелья, приготовлять всякие настои: умела и коров, и оленей лечить, умела и порчу насылать. Рассердится на кого-нибудь, вздумает ему навредить и как-то по-особенному поглядит своими злыми глазами на его дом. Смотришь: пройдет немного дней — дом сгорит. Глянет так же на соседскую корову — та молоко перестанет давать. Дети молока просят, плачут, а злюка только усмехается про себя, вот, мол, что я наделала. От одного её взгляда молоко сразу скисало, становилось невкусным, даже горьким.
	
	Никто не знал, из какой деревни пришла Ёма в эти края. Здесь познакомилась она с добрым охотником Иваном. Сумела хитрая Баба приколдовать его и поселилась е ним вместе в одинокой избушке. Но недолго царили здесь мир и согласие. Не могла Ёма жить спокойно. Часто покидала она жилище, чтобы наведаться в какую-нибудь деревню, навредить её жителям; иногда надолго углублялась в густую парму, вытаскивала добычу из чужих силков и канканов. В парме познакомилась она с Вэрсой, лешим.
	
	Злые люди рано стареют. Нехорошие мысли наложили отпечаток на лицо колдуньи. Нос у неё стал таким длинным, что нависал над верхней губой. Люди, заприметив издали колдунью, торопились обойти её стороной. Называть эту старуху стали Бабой-Ёмой и даже одним её именем пугали маленьких детей.
	
	Капризничает малыш, хнычет, спать не хочет, а то и встать утром ленится, мать его пугает:
	
	— Вовремя не встанешь или не ляжешь, или плакать не перестанешь — Бабу-Ёму позову. С большим мешком она придет, тебя заберет.
	
	Услышит малыш про колдунью и быстро успокаивается, слезы рукавом утирает, по сторонам оглядывается, спешит родительское распоряжение выполнить: боится Ёмы.
	
	Все удивлялись: как хороший человек, дед Иван, мог её терпеть? А он все сносил, старался образумить, чтобы поменьше зла она причиняла и уменье своё обращала на пользу людям, а не во вред.
	
	Но усилия пропадали даром. Баба даже грамоте учиться не захотела, а при её способностях она могла бы ей очень пригодиться. Колдунья веник в руки не брала, а наоборот, призывала в избу всяких жуков, тараканов и прочую нечисть. А иногда такую вонищу разведет колдовским варевом, что дышать становится нечем, стены и потолок покрываются жирными пятнами, а вокруг избы деревья начинают сохнуть.
	
	Кто знает, на сколько бы еще хватило терпения деда Ивана, если б в один зимний день, обходя силки, не услышал он детский плач, больше похожий на стон. Обернулся Иван и увидел на ёлке, возле занесенной снегом таёжной тропинки, маленькую девочку. Верно, в пути застигла её страшная беда: загрызли волки её родителей, а она, бедная, в страхе как-то вскарабкалась на дерево и, замерзая., сидела на нём.
	
	Бросился охотник к дереву, снял окоченевшую малютку, растёр снегом, закутал в свою одежду и быстро понёс домой.
	
	Думал старик, обрадует Бабу: детей-то у них самих не было. Думал, может, при виде малютки смягчится, сердце Ёмы, и станут они вдвоём растить сироту.
	
	Но чуть увидела Ёма девочку, затряслась от бешенства и заорала:
	
	— Что ты мне принёс?! Сейчас же отнеси её опять в лес! На что мне она?!
	
	Тут уж и дед Иван не выдержал: стукнул широкой ладонью по столу так, что изба ходуном заходила, а стол разлетелся в щепки. Ёма оробела, замолчала.
	
	Назвали сироту Марпидой. Тихая она была, не капризная. Но Баба исподтишка всячески старалась извести её. Уйдет дед на охоту, Баба по три дня сироте есть не дает, да еще побьёт ни за что ни про что. А как вернется дед, старуха ворчит, всякие нелепости наговаривает на малышку. Только дед не верил старой: знал её дурные привычки. А к девочке, послушной и ласковой, привязался старый охотник всем сердцем.
	
	Но заметил он, что Марпида худеет, бледнеет, чахнет: Ёма тайком её горьким зельем поила. Догадался об этом дед Иван, прикрикнул на Ёму, пристыдил её. А та взъярилась, схватила большой ковш, черпнула ядовитое зелье, которое как раз тогда варила, да и плеснула в лицо старому охотнику. Страшно закричал он ослепленный, так закричал, что весь лес откликнулся на тысячи голосов и услышали другие охотники, что случайно оказались поблизости, и отозвались. Испугалась Баба-Ёма, выскочила из избы и помчалась прочь в дебри таёжные, к Вэрсе. С тех пор долго её никто не видел.
	
	А дед Иван ослеп. Тяжело пришлось ему и маленькой Марпиде. Охотники вскоре ушли. Припасы кончались. Но тут, к счастью, выглянуло весеннее солнышко, потемнел и начал таять снег.
	
	Марпида оказалась на редкость умной и расторопной. Быстро научилась она силки на зверей и птиц ставить, рыбу ловить, сети чинить, пищу готовить, убирать в избе. Стало в ней чисто и уютно. Конечно, несмотря на все старания, девочка не могла сделать того, что делал дед, когда был зрячим, а потому они всё-таки жили бедно. Правда, понемногу он свыкся со слепотой, плел сети, выполнял кое-какую работу по дому, а главное, рассказывал Марпиде, что, где и как делать нужно. Многое знал и умел Иван, и многому научилась у него внимательная девочка. Подрастала она, с каждым днем умнела, хорошела и называла его отцом, батюшкой.
	
	В ту пору появился в тех местах знаменитый охотник Пера. Славился он силой и ловкостью, смекалкой и добротой. Как-то набрел он в лесу на одинокую избушку слепого, увидел юную Марпиду и стал им помогать.
	
	Полюбились молодому охотнику дед Иван и приветливая девушка. Нередко долгими зимними вечерами, когда за стеной гуляла Пурга, заходил в гости Пера посмотреть диковинные узоры, вышиваемые Марпидой, поглядеть, как она вяжет; слушал вместе с нею рассказы старика. От него узнал Пера о Книге мудрости и Камне жизни.
	
	Полагал старик, что чудесный Камень затерян где-то в снежных краях, и если люди найдут его, он принесет им счастье. А ведь Пера мечтал об этом. После рассказов деда Ивана молодой охотник уходил в парму, вслушивался в голоса птиц и зверей: не принесет ли кто из них весть о волшебном Камне. Но ничего не удалось выведать Пере. Отправился он по следам Ханонея. Полгода кружил по лесам и болотам, по рекам и тундре и опять-таки вернулся ни с чем. Не так-то легко было найти Камень жизни. Вы помните: Хановей и Пурга вдавили его глубоко в землю, засыпали ею, завалили снегом и льдом. Да и могла ли такая чудесная вещь лежать где-нибудь на виду?
	
	А Баба-Ёма, которая научилась не хуже охотников языку зверей и птиц, как-то услышала от болтливой сороки о Книге мудрости.
	
	— В ней написано, — верещала птица. — как найти волшебный Камень. Он и жжет и греет, и сжигает и оживляет, и лечит, и много чудес может сделать. А как им пользоваться, сказано в чудесной Книге, я сама её видела у Белой Совы. — затараторила лесная сплетница.
	
	Запомнились сорочьи слова Бабе-Еме. Поняла она, каким могучим станет обладатель волшебного Камня. Поняла она и то, что без Книги мудрости его не добыть, не узнать, где он спрятан. Решила Ема отыскать Книгу и завладеть Камнем.
	
	— Ох и много зла я смогу тогда сделать, — мечтала она, — разбогатею — и все будет мне подвластно — и реки, и леса, и люди, и звери. Всеми буду править я! Все будут мне служить и прислуживать.
	
	Вэрса тоже услышал о Камне жизни и Книге мудрости. Ему также захотелось добыть их, чтобы безнаказанно хозяйничать в лесном царстве и вредить добрым людям.
	
	Не сговариваясь, Баба-Ёма и Вэрса отправились разными путями на поиски Камня жизни к Великой северной равнине. Там мы с ними и познакомились, когда они попытались выманить Книгу мудрости у Белой Совы.
	
	\section[6. Призыв о помощи]{\center \textcolor{red}{6. ПРИЗЫВ О ПОМОЩИ}}
	
	\lettrine[findent=0pt]{\textbf{\textcolor{red}{К}}}{}огда Пера после безуспешных поисков пришел в избушку, Марпида вышивала полотенце. Заметила она, что Пера любуется её работой, и промолвила:
	
	— Тебе нравится — тебе и достанется. Закончу узор и дам тебе на память.
	
	Очень тронуло охотника внимание доброй девушки.
	
	Вдруг слепой приложил палец к губам. Все замолчали. В тишине слышалось только потрескивание горящих дров.
	
	— Не могу понять, что сегодня творится со мной? — тихо вымолвил дед Иван. — Почудилось, будто Она опять здесь и на меня оттуда, — он кивнул на угол за печкой, — с ненавистью смотрят её волчьи глаза.
	
	Марпида и Пера сразу догадались, что дед вспомнил о Ёме. Девушка побледнела, быстро подошла к печке, заглянула за неё:
	
	— Нет, батюшка, — прошептала Марпида, — никого здесь нет, да и не вернется Она сюда. И ведь с нами Пера...
	
	Девушка с надеждой посмотрела на охотника, тот слегка наклонил голову и улыбнулся.
	
	— Э-эх, хуже всего слепота, — протянул старик. — Сила еще есть, есть сноровка, уменье. А к чему они, когда не видишь, куда приложить их? Ничего не видишь...
	
	Девушка нежно прильнула головой к его плечу.
	
	Внезапно за стеной раздался собачий лай. Дверь отворилась. Худой мужичок в потрепанной одежде, едва переступив порог, шагнул к Пере, повалился ему в ноги и заголосил:
	
	— Выручи! Не дай погибнуть! Не дай помереть с голоду беспомощным детям! Только ты один можешь нас спасти!
	
	С мужичком вбежала большая лохматая собака, словно присоединяясь к мольбе хозяина, легла у ног Перы, и глядела на него ясными умными глазами.
	
	Пера поднял вошедшего и усадил на лавку. Марпида поставила перед ним еду и питье, дала поесть его лохматому спутнику. Мужичок понемногу пришел в себя и рассказал о том, что привело его сюда.
	
	Несколько дней назад возле его деревни появился страшный леший Вэрса и угнал всех коров, все стадо. Верная собака, которая предупредила о появлении вора, погибла от удара его тяжелой дубины. Немногочисленные жители деревни понимали, что сами не смогут одолеть могучего злодея. Они решили позвать на помощь Перу, и отправились во все стороны на розыски богатыря.
	
	— Кто еще справится с Вэрсой? На тебя вся надежда. Спаси нас и детей наших! — закончил грустный рассказ крестьянин.
	
	Уронил он голову на руки и заплакал. Дед и Марпида стали его успокаивать, а Пера задумался.
	
	Не боялся он лешего, никого не боялся, отдал бы свою жизнь, чтобы спасти других; всей душой хотел помочь беднякам. Но для этого нужно было найти Вэрсу. А легко ли увидеть лешего в его царстве? Он на дерево залезет — и не различишь его среди густых ветвей. Он сам в кует или дерево превратится — и ты мимо пройдешь, а он на тебя сзади набросится. Как же с ним встретиться?..
	
	Пера задумчиво потрепал по шее собаку. Та в ответ на ласку завиляла хвостом и доверчиво положила голову на колени богатыря.
	
	— Серко поможет тебе, — сказал крестьянин. — Он всюду поспеет и дорогу отыщет. Где мне, старому, не пройти, он пройдет. Возьми его себе на помощь.
	
	Крестьянин погладил собаку и что-то пошептал ей на ухо, указывая на Перу. Пес замотал хвостом.
	
	— Хотя бы коров отбить и вернуть беднякам, — подумал Пера; решительно поднялся, взял богатырскую палицу, лук и колчан со стрелами, подошел к крестьянину и попросил провести его к тому месту, где паслись коровы, когда нагрянул леший.
	
	— Оттуда, — решил Пера, — я пойду с собакой по его следам. `
	
	Лицо бедняка посветлело. Несмотря на усталость, он согласился сразу же пуститься в обратный путь. Марпида поспешно приготовила им на дорогу еду и вместе с дедом вышла проводить гостей.
	
	Сперва старик, держа под руку Марпиду, двигался довольно быстро, благо тропинка была знакомая и место ровное. Но через некоторое время, чтобы не задерживать Перу и его спутника, девушка и слепой решили вернуться и, не спеша, направились домой.
	
	А так как они идут медленно, мы задолго до их прихода успеем еще раз заглянуть в избушку, где познакомились с этими добрыми людьми.
	
	\section[7. Баба-Ёма хозяйничает]{\center \textcolor{red}{7. БАБА-ЁМА ХОЗЯЙНИЧАЕТ}}
	
	\lettrine[findent=0pt]{\textbf{\textcolor{red}{С}}}{}тарик не ошибся: за ним следили глаза Ёмы. Она не посмела приблизиться к двери, так как над нею висели большие щучьи зубы. А все древние коми знали, что это отпугивает всякую нечистую силу. Вэрса и Баба-Ёма это знали тоже и потому не решались заходить в те двери, над которыми висела раскрытая щучья пасть. Впрочем, Ёма не рискнула приблизиться к входу еще и потому, что её чуткое ухо уловило звуки незнакомого мужского голоса, доносившегося из избы. На всякий случай осторожная старуха подошла к дому с другой стороны, вынула из большой берестяной заплечной торбы, пестеря, как её называют коми, несколько длинных сосновых иголок, воткнула их в мох, чуть выступавший из-под одного бревна в стене, и шепнула колдовское слово. Бревно бесшумно отделилось от нижнего, на котором лежало, и чуточку приподнялось. В стене за печкой образовалась узенькая щёлочка. Ёма заглянула в неё, увидела Марпиду, Ивана, широкоплечего Перу и сразу догадалась, что это и есть тот знаменитый богатырь-охотник, о котором говорят в народе коми. Понимая, что в таком обществе ей делать нечего, Ёма бросила ненавидящий взгляд на слепого, и как раз этот взгляд он почувствовал. Но когда Марпида подошла к печке, послушные Ёме брёвна вновь неслышно сомкнулись, а сама колдунья, бормоча под нос разные ругательства, которые повторять неприлично, направилась в глубь леса. Однако не успела она пройти ста шагов, как услышала собачий лай.
	
	Старуха остановилась.
	
	Собака не обратила на неё внимания и пробежала мимо. Вслед за собакой появился запыхавшийся мужичок и вместе со своим хвостатым другом вошел в дом.
	
	— Это еще что за собрание?! — хмыкнула Ёма и хотела уйти подальше, но любопытство заставило её задержаться и последить за жилищем.
	
	Ждать пришлось недолго. Когда Пера и его спутники вышли из дверей, колдунья смекнула, что богатырь опять направляется в какой-то далёкий путь. Она поняла, что в избе никого не осталось, и едва голоса замолкли в отдалении, поспешно заковыляла к задней стенке дома. Там стала нашёптывать что-то и втыкать сосновые иглы между брёвнами. Подчиняясь колдовству, они начали раздвигаться. В стене образовался такой большой проём, что злодейка через него без труда проникла внутрь.
	
	— Фф-фу-у-уу — тяжело выдохнула она, когда очутилась посреди избы и огляделась.
	
	Всё здесь было ей не по нраву. Блестел тщательно вымытый и выскобленный пол. Лоснясь начищенными боками, на полочках стояли рядком глиняные горшки, деревянная и берестяная посуда.
	
	Ёма тоскливо поглядела на печку и стены, где не увидела даже ни одного таракана.
	
	— Всех вывела, — со злобой подумала старуха про Марпиду. — Всю живность истребила.
	
	Колдунья торопливо достала из заплечной торбы несколько больших мешков и принялась набивать их всем, что попадалось под руку. Желая не просто навредить добрым людям, а вообще сжить их со света, она заглянула в погреб, в чулан и сарай возле дома. Там хранились съестные припасы на долгую зиму. Грабительница не оставляла ни крошки, ни косточки, все складывала в мешки.
	
	Посуда не захотела перейти в руки Ёмы. Все глиняные горшки при её приближении разом грохнулись с полок и печки на пол и разбились на тысячи черенков. Один, самый пузатый горшок попытался свалиться прямо на злодейку, но неудачно перевернулся и уселся ей на голову, как шапка. Старуха его не разбила, а так и оставила на голове вместо горшка, разбитого при ссоре с Вэрсой из-за Книги мудрости. Все охотничьи и рыболовные снасти, стрелы, длинное копье с серебряным наконечником — всё захватила негодница. Что не могла взять — поломала. Особенно обрадовали её кремень и огниво.
	
	— Без них огня не добыть. Пусть замерзнут, — злорадно хмыкнула Баба-Ёма и затолкала находку в мешки. Заполнив их доверху, она самодовольно подбоченилась, любуясь опустошением, и прислушалась. Ничто не нарушало тишины.
	
	— А что если сжечь избу?! — мелькнуло в голове злодейки. — Но огонь и дым видны далеко и Пера и жильцы могут поторопиться с возвращением.
	
	— Нет. — нахмурилась Ёма, — пусть голодают в холодной избе. И она вылила на тлевший в печке огонь всю воду из берестяных вёдер. Затем Ёма с трудом протиснулась в дверь с награбленным добром и бросила торжествующий взгляд на щучью пасть, которая могла помешать только войти в дом, но не выйти из него.
	
	Быстро темнело. Того и гляди, вернутся хозяева, — подумала воровка. Уже за порогом она накрепко связала друг с другом узлы, столкнула в реку все вёдра, кадки, все заготовленные на зиму дрова и подошла к высокой берёзе у дома.
	
	Что нашёптывала дереву колдунья, неизвестно. Но, подчиняясь заклинаниям, оно низко наклонилось. Ёма привязала к крепкому суку на вершине ствола огромный ком награбленного, отошла на почтительное расстояние, что-то пробормотала и протянула руки в том направлении, откуда пришла.
	
	Внезапно дерево резко распрямилось, мешки и узлы сорвались с его вершины и понеслись по воздуху туда, куда указывала колдунья. Она довольно крякнула, трижды плюнула через правое плечо на опустошенную избу и с неожиданным проворством быстро зашагала через лес к той поляне, где условилась встретиться с Вэрсой.
	
	\section[8. Возвращение]{\center \textcolor{red}{8. ВОЗВРАЩЕНИЕ}}
	
\lettrine[findent=0pt]{\textbf{\textcolor{red}{Г}}}{}устые тучи опускались на верхушки деревьев. Сырой осенний ветер дул в лица старика и Марпиды. Оба устали и проголодались. Уже стемнело, когда они подошли к дому.
	
	— Хорошо бы сейчас погреться и поужинать... горяченького... — заметил слепой.
	
	Марпида не ответила.
	
	— Странно, — подумала она, сразу приметив открытую настежь дверь.
	
	Едва они переступили порог, как пахнуло сыростью: Ёма залила огонь в печке и расплескала воду по полу.
	
	Дед Иван пожал плечами в недоумении, затем попытался осторожно приблизиться к печке. Под ногами затрещали глиняные черепки от разбитой посуды.
	
	— Кто тут хозяйничал?.. — дед остановился, вытянул вперед руки, сделал неуверенно шаг, другой, коснулся ладонями холодной печки и тихо сказал; — Надо поскорее развести огонь.
	
	Слепой шарил руками по печке, но кремня, огнива, трута и бересты на обычном месте не находил.
	
	Глаза Марпиды уже привыкли к темноте. Девушка спустилась в погреб, обшарила чулан, вышла во дворик — и поняла все: их обворовали; даже дрова и те исчезли. У Марпиды подкосились ноги. Она бессильно прислонилась к высокой берёзе. Резкий порыв ветра ударил в лицо девушке.
	
	— Марпида, где ты?! — услышала она голос старика и подняла глаза. Над её головой колыхалось что-то белое. Это было полотенце, то самое, которое она с такой любовью вышивала для Перы. Марпида осторожно сняла е ветки эту единственную, случайно уцелевшую вещь и вошла в дом.
	
	Нечем было добыть огонь, нечего было приготовить даже на самый скромный ужин.
	
	— Батюшка, милый, что мы будем делать? — сквозь слезы сказала девушка, обнимая старика. — Какой-то злодей ограбил нас дочиста, ни сушеного грибка, ни зёрнышка, ни ягодки не оставил!
	
	Слепой гладил мягкие волосы Марпиды. Что он мог сказать? Чем мог утешить? Пера ушел надолго. Надеяться было не на кого и не на что.
	
	За разбитым окном всё яростнее гудел холодный ветер, и снег серебряным покрывалом ложился на землю. Начиналась зима.
	
	\section[9. Жирные мыши и Книга мудрости]{\center \textcolor{red}{9. ЖИРНЫЕ МЫШИ И КНИГА МУДРОСТИ}}
	
	
	\lettrine[findent=0pt]{\textbf{\textcolor{red}{В}}}{}эрса загнал коров в глухое урочище, напился вдоволь молока и бросил стадо, а сам побежал к месту свидания с Ёмой. Вскоре он очутился на берегу довольно широкой реки. Перепрыгнуть через неё леший не отважился. Переходить вброд он тоже не хотел, потому что не ладил с повелителем рек и озер Ва-Кулем. Возможно, из-за этого леший никогда не умывался, его тело покрылось густым слоем грязи, спина и грудь заросли мохом.
	
	Вэрса раздумывал, как переправиться через реку, пока не заметил в зарослях ивняка, недалеко от себя, лодку-долблёнку. На дне её лежали сети и весло. Обрадованный Вэрса вскочил в неё, оттолкнулся от берега и переправился на другую сторону. Неизвестного хозяина лодки он отблагодарил тем, что выкинул за борт сети, забросил далеко в лес весло, а ни в чем не повинную лодку перевернул и вытолкнул на волю течения, в самую стремнину. Эта проделка привела лешего в такой восторг, что, добравшись до места встречи с Бабой-Ёмой, он в ожидании подруги начал кувыркаться и, наконец, опустился на четвереньки и захрюкал от удовольствия так громко, что испуганные вороны с тревожным карканьем вылетели из гнезд и полетели прочь…
	
	Неизвестно, сколько еще времени продолжал бы он подобные занятия, если б сильный удар чем-то большим и мягким по нижней части спины не заставил безобразника уткнуться носом в землю.
	
	Рядом шлёпнулось еще что-то и еще... Леший скосил единственный глаз и увидел несколько больших узлов и мешков. Они явно свалились откуда-то сверху. Вэрса прыгнул под ближайшее густое дерево и с опаской посмотрел на небо. Больше ничего не падало. Однако он оробел и решил дождаться наступления темноты, чтобы ближе познакомиться с таинственными мешками. Он присел под сосной и не спускал с них глаза: а что если снова улетят?..
	
	Вдруг ветки кустов раздвинулись. На поляну вышла Баба-Ёма. На её голове красовался новый глиняный горшок, на шее, как ожерелья, болтались связки сушеных грибов, за поясом торчали топор, пара ножей, несколько деревянных ложек. Через плечо колдуньи были небрежно перекинуты шкурки лисиц вперемежку с вяленой рыбой, а на ногах пестрели мягкие сапожки из оленьих шкур.
	
	Опиралась Ёма на копьё с блестящим серебряным наконечником. Она удовлетворённо глянула на мешки и узлы: все в целости. Только теперь леший догадался, откуда они взялись, и подскочил к старухе.
	
	— Ну? Что? Откуда? У кого? — посыпались отрывистые вопросы.
	
	— У слепого, — с недоброй усмешкой ответила Ёма и стала развязывать узлы.
	
	Глаз лешего разгорелся: кроме охотничьих и рыболовных снастей, мехов, одежды и украшений, к которым он был равнодушен, прожорливый Вэрса увидел множество съестных припасов. Дрожа от жадности, он сразу набросился на пищу, заталкивая в огромный рот оленьи ножки, солёную рыбу с костями, сушеные грибы, сладкие ягоды... Только желудок лешего мог переварить подобную смесь.
	
	Напрасно подруга пыталась удержать ненасытного приятеля. Наконец, он наелся и, тяжело дыша, сел на пенёк, поглаживая руками раздувшийся живот.
	
	Старуха попросила помочь ей унести и припрятать награбленное. Но леший отказался. Зная, что спорить с ним бесполезно, Ёма в сердцах плюнула и сама принялась укладывать добычу.
	
	Вдруг Вэрса легонько толкнул подругу в бок и указал грязным пальцем куда-то вдаль, поверх сосен. Там, над их верхушками, появилось маленькое пятнышко. Приближаясь, оно всё увеличивалось.
	
	Ёма подняла голову:
	
	— Она, — прошептала колдунья.
	
	— Тяжело летит, — заметил Вэрса. — Устала.
	
	— Заманить бы. — прогундосила Ёма. — Вдруг при ней…
	
	Она знаком приказала хранить молчание, приложила ладони рупором ко рту и, подражая голосу совы, закричала:
	
	— У-гу-у! У-гу-у! Лети сюда, Белая Сова!.. Не сердись на нас! У нас есть вкусные жирные мыши. Милости просим в гости, Белая Сова!..
	
	Действительно, над лесом летела наша старая знакомая. Получив приказ доставить Книгу мудрости на Тэв-Поз-Из, она отлетела подальше от Пурги и Хановея, а потом, зная, что они еще надолго задержатся в тундре, наведалась к своим пернатым родственникам. В их обществе Сова посетовала на своих неотесанных хозяев, немного отдохнула, нежно поцеловала чужих птенцов, простилась и полетела дальше. В гостях она почти не ела, только попробовала пару вкусных ягод, заготовленных на зиму.
	
	В ожидании морозов лее обнажился, и птица с небольшой высоты могла его хорошо просмотреть. Увы, никакой добычи для себя, даже захудалого зайчишки, Сова не заметила. И тут она услышала крик Ёмы. Осторожная птица была невысокого мнения о колдунье и её спутнике, которые вздумали клянчить у неё Книгу мудрости. Но... известие о жирных мышах, её любимом блюде, заставило Сову позабыть о недавнем приключении.
	
	— А кто их знает? — подумала она. — Что если от чистого сердца хотят угостить, благо сами мышей не едят. Может, их совесть мучает за то, что так грубо пытались выманить у меня Книгу?.. Да и что они могут мне сделать, если не умеют ни читать, ни летать?.. А, может, лучше поостеречься?..
	
	Но голос благоразумия заглушило урчание пустого желудка и, ради жирных мышей, хранительница Книги мудрости опустилась на поляну.
	
	Она поклонилась Ёме и Вэрсе и вопросительно посмотрела на них: где мыши? Колдунья сразу поняла взгляд птицы и засуетилась:
	
	— Сейчас, сейчас!
	
	При этом хитрая старуха поспешно сняла с себя одно ожерелье из сушеных грибов и положила на землю. Сова была не прочь их попробовать, но еще больше её интересовали мыши.
	
	— Что там у тебя под крылом, Книга мудрости? — как бы невзначай спросила Ёма, снимая с нитки гриб и любезно поднося его к самому клюву птицы.
	
	— У-гу, — утвердительно кивнула Сова и слегка отодвинулась от чрезмерно внимательной Бабы-Ёмы.
	
	— Ах, — старуха скроила умильную рожу, — хоть бы посмотреть поближе, полюбоваться ею.
	
	— Нельзя, — сухо возразила Сова. — А где мыши?
	
	— Есть! — воскликнула Ёма. — Есть и будут мыши! Кстати, сколько мышей стоит такая книга? — полюбопытствовала она, стараясь заглянуть под крыло.
	
	— Книга мудрости стоит дороже мышей, — многозначительно заметила Сова.
	
	— Дай-ка мне подержать её!
	
	С этими словами нетерпеливый леший неожиданно ухватился за уголок переплета, торчавший из-под крыла. Одновременно Ёма, как пружина, резко выпрямилась, подалась вперед и навалилась на птицу. Та попыталась поднять крылья, чтобы взлететь, но за ремешок, на котором была Книга, успела ухватиться колдунья. Сова ударила её своим страшным клювом. Злодейка пошатнулась и упала. Но тут же вскочила на ноги и бросилась к Сове с пустым мешком, стараясь накинуть его на голову птице. Та увернулась и снова попыталась взлететь. Тут Вэрса вцепился в неё. Сова клюнула и его. Он дико вскрикнул и, забыв о Книге, первой попавшейся под руку веткой ударил птицу по крылу в то самое мгновение, когда она, собрав все силы, оторвалась от нападающих и взлетела.
	
	— Далеко не уйдет: я ей крыло перебил. — злобно прохрипел вслед Вэрса, держась за рассеченное клювом плечо. Вместе с Ёмой он попытался преследовать Хранительницу, но вскоре махнул рукой.
	
	Сова скрылась за деревьями.
	
	Преследователи вернулись.
	
	— Все ты виноват: поторопился. — буркнула Ёма.
	
	Вэрса не возражал. Они потащили мешки с награбленным в глубь леса и решили на другой день хорошенько осмотреть чащу в надежде найти Книгу мудрости и тяжелораненую Сову.
	
	На полянке, где за короткое время произошло столько бурных событий, воцарилась тишина.
	
	Кряхтя под тяжестью ноши, Баба-Ёма и Вэрса брели по лесу. Они уже порядочно прошли, когда Баба-Ёма вдруг остановилась и спросила спутника:
	
	— А копьё?
	
	Вэрса никак не мог вспомнить, где он его оставил. Разгорелся спор. Он закончился тем, что леший переложил весь груз на спину подруги и отправился в обратный путь разыскивать копьё, к уже знакомой нам поляне.
	
\section[10. Новые события на маленькой поляне]{\center \textcolor{red}{10. НОВЫЕ СОБЫТИЯ НА МАЛЕНЬКОЙ ПОЛЯНЕ}}
	
	\lettrine[findent=0pt]{\textbf{\textcolor{red}{Ч}}}{}еловек читает глазам, собака — носом. Он её никогда не обманет. Из тысячи различных запахов она умеет выбрать один, особенно нужный.
	
	Пера и его спутники добрались до далёкой деревни, но заходить в неё не стали. Старому хозяину уже не под силу были большие переходы, он ткнул Серко несколько раз носом в коровьи следы, потом указал на Перу и поплелся домой.
	
	Пёс догадался, что временно его передают в распоряжение нового хозяина и утвердительно вильнул хвостом: согласен.
	
	С широкого луга коровьи следы вели в чащу. Недавно прошёл дождь, смешанный со снегом, что затрудняло поиск.
	
	Серко внюхивался в чуть примятую траву. Хотя пса не знакомили с математикой, он твердо знал, что у каждого нормального существа четыре ноги. Люди не составляли исключения: просто они привыкли ходить на задних лапах. Коровы оставляли четыре следа, птицы — два смешанных следа, похожих на царапины. Любого зверя, любую птицу собака безошибочно различала по виду следа и, главное, по запаху.
	
	Со времени исчезновения коров прошло уже несколько дней, но чуткий пёс отыскал их следы. Однако среди них его смутили еще два следа. Один походил на след копыта, был больше коровьего, хотя меньше лошадиного, другой — на человеческий, но был значительно крупнее. Оба следа все время были рядом и имели один запах, принадлежали одному существу. Какое же оно? Почему оно следовало повсюду за коровами? Не оно ли угоняло их? Следов было много. Зайцы, медведи, росомахи, лоси, волки, барсуки, белки, лисицы, ласки, горностаи — звери и птицы, которыми так богаты северные леса, — все оставляли здесь свои отпечатки. Но собака, замечая их, шла только по тем, которые требовались сейчас — по следам коров и странного существа.
	
	Серко очень нравился новый хозяин и пёс про себя жалел, что не имеет права нарушить закон собачьей верности, а то бы навсегда ушёл от своего старого ворчуна к Пере. Но собаки не выбирают себе хозяев, и Серко продолжал своё дело.
	
	Пера понял, что может довериться умному животному, и спокойно шел за ним. Иногда Серко подбегал к молодому хозяину, затем — к странным следам, и охотник догадывался, что пёс чует похитителя.
	
	Наши герои уже углубились далеко в чащу, когда пёс начал выказывать признаки беспокойства: следы разделялись. Коровьи вели еще дальше в лее, следы загадочного существа — к небольшой поляне. Её наполняли тревожные запахи.
	
	Начинало темнеть, и Пера глазами не мог различить на земле то, что на ней читал носом Серко. Здесь собаку поражало обилие человеческих запахов, даже будто знакомых, будто запахов того уютного домика, в который его привел старый хозяин к Пере. И другие запахи... Среди них — загадочного похитителя коров.
	
	Серко взволнованно закружил по поляне: здесь произошло нечто нехорошее: он почуял запах крови.
	
	След большой птицы... еще след и кровь... кровь… Под сломанным кустом сквозь голые ветки просвечивало что-то золотистое. Там лежал большой лист необыкновенной, очень плотной бумаги со светящимися буквами. И на ней алело пятно свежей крови.
	
	Пёс ощетинился, хотел тявкнуть., но раздумал и посмотрел на хозяина. Тот внимательно следил за собакой; подошёл, нагнулся и уже хотел поднять лист, как . вдруг Серко отпрыгнул в сторону и залаял. Пера быстро обернулся. Собака, дрожа от ярости и страха, рычала на огромный трухлявый пень, которого они раньше не заметили.
	
	Пера улыбнулся и хотел снова наклониться за бумажным листом, но Серко пронзительно взвизгнул, бросился к охотнику и прижался к его ногам. Пера схватил лист и выпрямился. В нескольких шагах от себя богатырь увидел рослого плечистого мужчину. Он стоял перед пнем, полностью заслоняя его собой. Сгущавшиеся сумерки не позволяли рассмотреть лицо незнакомца.
	
	— Ты кто?! — вместо приветствия грозно окликнул он Перу, косясь единственным глазом на рычащую собаку. — Уйми пса!
	
	И он замахнулся на Серко огромной суковатой палкой.
	
	— Чего пришел?! Тут мои владения!
	
	Умный человек не позволит себе грубости. Обращение неизвестного не столько рассердило, сколько позабавило Перу.
	
	— А кто тебе дал эти владения? — спокойно спросил он.
	
	— Как кто? — вспыхнул грубиян. — Мои — и всё тут!
	
	— Нет, не всё. А где бумага? — не повышая голоса, продолжал Пера. — Теперь без неё никто никаких прав не имеет.
	
	— А у тебя она есть, что ли? — уже тише спросил незнакомец, смерив его взглядом.
	
	Вместо ответа Пера выставил перед собой лист с золотистыми буквами:
	
	— Вот, читай княжескую грамоту. Это не твои, а мои угодья, и моих земляков.
	
	Спокойная уверенность богатыря смутила незнакомца.
	
	— Я читать не собираюсь, не умею и не буду. — заметил он. — Так что грамота — мне не указ. Кто сильнее, тот и владеет лесом. Может, силами со мной померяешься?..
	
	В словах прозвучала угроза. Грубиян на шаг отступил, цепкой рукой схватил высокую молодую березу, вырвал её из земли, как травинку, быстро обломал корни и ветви и протянул ствол одним концом охотнику:
	
	— Держи. Кто перетянет, тому и владеть лесом.
	
	Пера сунул за пазуху найденную бумагу и кивнул в знак согласия.
	
	Серко догадался, что тут его помощь не нужна. Правда, он не мог понять, почему его молодой хозяин так долго разговаривает с незнакомцем. Пёс успел  заметить, что тот возник из страшного трухлявого пня, который исчез, чуть только грубиян появился перед Перой. А главное, что скрывала густая трава, это ноги неизвестного: одна из них была вроде копыта. Их следы так смущали собаку во время поисков стада. Серко понял, что перед ним Вэрса. Но... требовалось соблюдать приличие и, не спуская глаз с лешего, пёс стал в сторонке, готовый в любое мгновение броситься на выручку своему хозяину.
	
	Пера и Вэрса крепко ухватились за концы берёзового ствола, выше чем по щиколотки вдавили ноги в землю и старались перетянуть друг друга. Иногда леший резко вскрикивал, пытаясь испугать соперника и перетянуть на свою сторону.
	
	Вдруг в берёзе что-то хрустнуло.
	
	— Что трещит? — отрывисто спросил Вэрса.
	
	— Твой хребет, — ответил Пера.
	
	Леший оробел: а вдруг это его горб треснул от натуги.
	
	— Хватит, — сказал он. — Вижу, ты крепок. Давай жить дружно.
	
	Оба одновременно выпустили из рук концы ствола и шагнули друг другу навстречу.
	
	— Как звать-то? — уже миролюбиво спросил леший.
	
	— Перой.
	
	При этом имени Вэреа разом вспомнил предостережения Ёмы и слухи о невиданной силе прославленного охотника.
	
	— А тебя как звать? — поинтересовался Пера.
	
	— Никак, — отрезал Вэрса. — У меня нет имени. — И он потянулся к палке.
	
	Серко зарычал. Вэрса сообразил, что собака не позволит ему незаметно подкрасться к сопернику или каким-нибудь другим способом внезапно напасть на него.
	
	В свою очередь охотник, приглядываясь к собеседнику, все более убеждалея, что перед ним какое-то необычное существо, по приметам очень похожее на Вэрсу.
	
	Одноглазый глянул на пестерь охотника и предложил разжечь нодью, негаснущий костер и поужинать.
	
	Свалили сосну. Вскоре затрещал веселый огонь, запахло жареной зайчатиной, и несколько жирных костей немного успокоили проголодавшегося Серко.
	
	После трапезы охотник накрылся плащом и улегся по одну сторону костра, леший — по другую. Но Пера не уснул. Уже почти уверенный, что имеет дело с Вэрсой, богатырь не забывал о бедных земляках, думал о коровах: где они? Как их найти? Он считал, что допытываться о них не стоит: Вэрса навряд ли скажет правду. Скорее всего, они поблизости. Утром Серко снова наведет его на следы. Если же сейчас выспрашивать Вэрсу, тот может незаметно скрыться в темноте, убежать и угнать все стадо еще дальше.
	
	Все же Пера было сомкнул веки, но тут же почувствовал на лице холодок: влажным носом Серко ткнулся в его щеку. Пера открыл глаза: Вэрса исчез. Богатырь тихонько отодвинулся от костра, стащил с себя плащ, прикрыл им лежавшее рядом бревно и отполз в темноту.
	
	Лес шумел. Ветер усиливался. Его порывы ломали старые сухие деревья, и они с громким треском рушились на землю. Темнота полнилась тысячами звуков.
	
	Внезапно порыв ветра донес отдаленное мычание коров. Пера положил руку на голову навострившего уши пса. Тот, наверное, еще раньше услышал мычание, но почему-то сейчас напряженно смотрел в другую сторону. Пера тоже глянул туда и поспешно зажал рукой пасть мохнатого друга: в нескольких шагах от костра появился Вэрса. Он держал в одной руке огромный камень, в другой — копьё с блестящим наконечником. Леший осторожно прислонил его к кусту, двумя руками поднял камень над головой и с бешеной силой швырнул в то место, где полминуты тому назад лежал Пера.
	
	Вслед за камнем леший метнул копьё. Оно воткнулось в накрытое плащом бревно, глухо загудевшее от могучего удара.
	
	— Ну и крепок был Пера! — заорал обрадованный негодяй, подбегая к копью и выдергивая его. — Ох и крепок!
	
	— Да еще как крепок! — воскликнул подоспевший Пера и богатырской палицей ударил опешившего злодея по горбу.
	
	Собака вцепилась лешему в ногу. Он растерялся, завопил и бросился в чащу. Серко и Пера побежали за ним. Пере удалось догнать его и схватить за руку. Раздался сухой треск, сопровождаемый воплем, и в руке охотника вместо руки Вэрсы оказалась толстая ветка. В ту же минуту посветлело: из-за туч выглянула луна. Между деревьями показался раненый леший. Он вырвал с корнем и бросил под ноги преследователю толстую пихту, а сам скрылся в густых зарослях.
	
	Вдруг у самого уха богатыря со свистом пронеслось копьё, а Серко с яростью вцепился в кору стоявшего неподалеку дерева. Из-под нее брызнула кровь. Пера ударил по стволу палицей. Она переломилась пополам. Дерево застонало, тяжело рухнуло, пытаясь раздавить охотника, а затем вдруг само поднялось... и превратилось в Вэрсу. Выкрикивая угрозы и крича от боли, он бежал к поляне. У самой опушки Пера настиг его… и споткнулся: вместо врага перед охотником вырос чёрный пень. А из-за него, прихрамывая, выскочил громадный раненый волк, щелкнул зубами и прыгнул на грудь Пере. Но богатырь успел всадить острый охотничий нож в сердце зверя. С воем покатился он по поляне и замер. Когда Пера и Серко подбежали к нему — все было кончено: вместо волка у еще горевшего костра лежал мертвый Вэрса.
	
	\section[11. Найденная страница]{\center \textcolor{red}{11. НАЙДЕННАЯ СТРАНИЦА}}

	
	\lettrine[findent=0pt]{\textbf{\textcolor{red}{П}}}{}ера вытер со лба пот, нежно погладил Серко и, чтобы немного остыть, начал снимать охотничий лаз, так называют коми кожаную безрукавку, одеваемую через голову. Тут на груди его зашелестела найденная бумага. Он вспомнил о ней и достал её из-за пазухи.
	
	На бумаге светились большие выпуклые буквы:
	
	
	
	
\qquad \qquad \qquad \qquad 	Только сильный путь осилит,

\qquad \qquad \qquad \qquad 	Только смелый не отступит.
	
	
	
	
	Эти слова посередине листа сразу бросались в глаза, но вырванная или выпавшая из какой-то книги страница начиналась не с них, а с окончания фразы:
	
	
	
	
\qquad \qquad \qquad \qquad 	Нужно сжечь на Камне жизни
	
\qquad \qquad \qquad \qquad 	И ещё горячий пепел
	
\qquad \qquad \qquad \qquad 	Приложить к дрожащим векам:
	
\qquad \qquad \qquad \qquad 	Тот, кто зрения лишился,
	
\qquad \qquad \qquad \qquad 	Широко глаза откроет.
	
\qquad \qquad \qquad \qquad 	Все увидит, всех узнает.
	
	
	
	
	Сердце богатыря радостно забилось. Он понял, что держит в руках страницу из Книги мудрости. Ведь только в ней говорилось о Камне жизни. Далее перечислялись многие удивительные свойства таинственного Камня, е помощью которого можно исцелять болезни, возвращать зрение даже тем, кто сроду не видел. Узнал Пера, как, положив его в очаг, осчастливить дом, в котором он горит, не сгорая.
	
	Но как отыскать его?
	
	Страница завершалась обрывком фразы все о том же Камне:
	
	
	
	
\qquad \qquad \qquad \qquad 	Путь откроется тому лишь
	
\qquad \qquad \qquad \qquad 	С Тэв-Иоз-Из крутой вершины…
	
	
	
	
	Окончание, наверное, было на следующем листе. Пера стоял в раздумье и повторял про себя красивые слова из чудесной Книги. Мысленно он благодарил деда Ивана за то, что старик, сам будучи слепым, сумел научить его грамоте. И, хотя Пера познакомился всего с одним листком из Книги, но уже почувствовал себя богаче и увереннее: значит, есть и Книга мудрости и Камень жизни. Значит, можно и нужно их найти во что бы то ни стало, чем скорее, тем лучше — и дать счастье людям. Пера вновь и вновь внимательно вчитывался в написанное:
	
	
	
	
\qquad \qquad \qquad \qquad 	Путь откроется тому лишь
	
\qquad \qquad \qquad \qquad 	С Тэв-Поз-Из крутой вершины...
	
	
	
	
	Ясно: первым делом следует отправиться на Тэв-Поз-Из!
	
	Очень хотелось поскорее пригнать в деревню коров. Их настойчивое мычанье слышалось все ближе. Хотелось с листом из Книги мудрости побежать к деду Ивану и Марпиде. Однако Пера решил, что самое лучшее немедленно отправиться на вершину Тэв-Поз-Из: может быть, там удастся познакомиться с другими страницами Книги мудрости или даже найти Камень жизни.
	
	— Что толку, — рассуждал Пера, если я е одной страницей приду к людям? В ней нет ни начала ни конца. Нужно найти и прочесть по порядку всю Книгу, иначе её не понять.
	
	— Я обрадую Марпиду и деда, — размышлял он. — Вернусь целым и невредимым с известием о победе над Вэрсой; покажу найденный лист из Книги мудрости. А какую пользу я принесу?
	
	Пера не подозревал о несчастье, которое за короткое время его отсутствия обрушилось на жителей одинокой избушки.
	
	Ему очень хотелось повидать Марпиду. Но нельзя было терять время. Если, напав на след чудесной Книги, он не поторопится её найти сейчас, позже это станет делом ещё более трудным или вовсе неосуществимым. В чьих руках Книга? Ведь не случайно найденная страница очутилась здесь, в царстве Вэрсы. Вероятно, кто-то уже хотел завладеть Книгой — и завладеть силой. А. кто?.. А если какой-либо злодей проникнет в сокровенные тайны Книги и добудет Камень?..
	
	Рассвело. Звонкий лай вывел богатыря из раздумья. На поляне показались коровы. Через несколько минут собралось все стадо; ни одна не пропала. Весело помахивая хвостом, выбежал Серко. Пера обнял его, погладил по умной голове и, указывая в сторону деревни и на коров, несколько раз повторил:
	
	— Домой, Серко, домой. Всех домой!
	
	Собака бросилась исполнять приказание. Вскоре её лай и мычание коров затихли вдали.
	
	А Пера зашагал в противоположную сторону. Предстоял путь через леса и болота, озёра и реки, к Большим камням, к вершине Тэв-Поз-Из. Но Пера не думал о трудностях, не обращал внимания на усилившийся ветер, на снег, летевший в лицо. Мечталось о Камне жизни, о Книге мудрости, которая поможет всю чудесную силу волшебного Камня обратить на радость и счастье людям.
	
	Богатырь быстро шёл по лесу. Не догадывался он, что позади, скрежеща зубами от злости, брела Баба-Ёма. К утру ей надоело ждать Вэрсу. Она отправилась к поляне и, ещё не дойдя до неё, нашла копьё. Выйдя на поляну, Ёма увидела труп лешего и все поняла. На свежем снегу она легко отыскала следы Перы и поторопилась за ним, чтобы отомстить за смерть Вэрсы.
	
	
		\section[12. Мучительный полёт]{\center \textcolor{red}{12. МУЧИТЕЛЬНЫЙ ПОЛЁТ}}
	
	
	\lettrine[findent=0pt]{\textbf{\textcolor{red}{В}}}{} первые мгновения, когда Сове удалось вырваться из цепких рук Вэрсы и взлететь, она забыла о своей ране. Но с каждым взмахом крыльев боль усиливалась и, наконец, стала нестерпимой. Перья левого крыла отяжелели и слиплись от крови. Книга под крылом казалась огромным грузом. Сова выбивалась из сил. Спуститься на землю она боялась, страшась преследования. Ведь Ёма и Вэрса действительно некоторое время бежали за ней. Кроме того, птица опасалась, что, очутившись на земле, уже не сможет собраться с силами для взлёта. А лететь становилось все труднее. Мелкие и частые хлопья мокрого снега налипали на хвост. Разгулявшийся ветер относил Сову куда-то в сторону.
	
	Так прошла ночь. Весь следующий день птица тоже провела в полете. Лишь изредка она решалась на несколько минут присесть на каком-нибудь каменистом бугорке, чтобы перевести дыхание, и снова продолжала путь к вершине Тэв-Поз-Из.
	
	Близилась вторая бессонная ночь. Усталая птица летела почти бессознательно, порою вслепую, закрывая глаза от боли; летела, потеряв направление, то опускаясь к оголенным веткам деревьев, то взмывая над ними, чтобы вскоре снова опуститься к самой земле. Один раз во время такого невольного спуска Сова чуть не попала в зубы голодному волку. Еле-еле оторвавшись от хищника, она заметила, что кто-то упорно следует за ней по земле. Сова снизилась и узнала лису. Преследовательница ожидала, когда обессиленная птица упадет. Капли крови, все чаще падавшие на снег, указывали лисице нужное направление. Изредка она поднимала острую мордочку, облизывалась, глядя на Сову, и бежала за ней.
	
	Сгустившаяся тьма не принесла облегчения. Ночью похолодало. Крылья отяжелели, утратили гибкость и упругость. Вопреки обычаям своих предков, никогда не садившихся на деревья, птица перелетала с одной сосны на другую, судорожно цеплялась за ветку, переводила дух, опять взлетала на две-три минуты и опять отдыхала. Перелеты становились все короче, передышки — все дольше.
	
	Метель не унималась. Вихри снега слепили Сову. Надеясь увидеть, хотя бы вдалеке, цепи гор и скалу Гэв-Поз-Из, она попыталась взлететь повыше, подняться над нависшими тучами. Превозмогая боль, Сова резко взмахнула крыльями, рванулась вверх, но тут в её глазах завертелись разноцветные снежинки, закружились ветки, опрокинулись сосны, и, теряя сознание, птица слабо вскрикнула и упала на снег.
	
	
	
		\section[13. Нежданная гостья]{\center \textcolor{red}{13. НЕЖДАННАЯ ГОСТЬЯ}}
		
	
	\lettrine[findent=0pt]{\textbf{\textcolor{red}{М}}}{}арпида не напрасно обшаривала весь двор и чердак. Она нашла сломанный нож, старую оленью шкуру, несколько сушёных грибов и лаже куеок чёрствого мёрзлого хлеба.
	
	Обрадованная девушка поспешила с находками в избу.
	
	Дед Иван, дуя на озябшие пальцы, прилаживал дверной засов. Занятые работой, дед и Марпида меньше чувствовали голод и холод: дела отвлекали от мыслей о пище.
	
	Уже несколько суток они ничего не ели, кроме сушёных ягод и грибов, и очень ослабли. Найденный ломоть хлеба был как нельзя кстати.
	
	— Он большой. — говорила девушка. — Нам обоим сегодня хватит на ужин. Она старалась подбодрить себя и деда, даже стала напевать песенку и бойко притоптывала ногами, чтобы согреться.
	
	Старик приладил засов, сел на лавку, взял два камня, принесенные Марпидой, выдернул из стены немного сухого мха, положил его рядом с собой и, ударяя камень о камень, напрасно пытался высечь искру, чтобы поджечь мох и развести огонь в печке.
	
	Марпида заботливо укутала ноги деда оленьей шкурой, а сама снова затопала вокруг стола, прижимая к груди замерзший кусок хлеба. Когда он заметно оттаял, девушка положила его на стол, подышала на него и торжественно объявила:
	
	— Готов. Пора ужинать.
	
	— Ешь сама на здоровье, — улыбнулся старик. На ночь глядя, я не привык наедаться.
	
	Девушка поняла: он хочет, чтобы она съела весь кусок, и стала убеждать не отказываться от еды.
	
	Наконец, он согласился, но с условием: разделить хлеб точно пополам.
	
	Девушка обломком ножа стукнула по краю ломтя. Он хрустнул.
	
	— Вот. — сказала Марпида, — поровну разломила, разделила. Говори: какой кусок тебе, какой мне?
	
	Слепой засмеялся, протянул ладонь и девушка положила на неё весь кусок.
	
	— Точно половина. — деловито заметила она, чтобы скрыть обман: себе она не взяла ни крошки.
	
	Старик тяжело вздохну, взвешивая хлеб на ладони, и хотел уже поднести его ко рту, но вдруг опустил руку:
	
	— Что это?
	
	В шуме разгулявшейся непогоды чуткое ухо слепого различило какие-то необычные звуки.
	
	Оба прислушались.
	
	— А, может, это Ёма... или Вэрса? — шёпотом, еле шевеля губами, выдавила Марпида.
	
	— Опять, — старик кивнул на дверь. За нею раздался жалобный стон... один... другой…
	
	Забыв о страхе, думая лишь о том, что кому-то сейчас плохо и нужно помочь, девушка выбежала из дома.
	
	Слепой тоже быстро шагнул к двери и ступил за порог.
	
	— Марпида! — окликнул он. — Где ты? Кто там? Марпида!
	
	— Вот я! — запорошенная снегом, она появилась совсем рядом, будто вынырнула из сугроба. В руках у неё было что-то очень большое и грузное.
	
	Они вошли в дом, и девушка склонилась над принесенным пушистым комом. Он чуть заметно шевелился и изредка тихо стонал.
	
	— Кто это? — спросил слепой.
	
	— Белая Сова, очень большая и красивая.
	
	Марпида осторожно стала сметать снег с её перьев и дышать на её голову и грудь.
	
	Сова опять застонала, открыла глаза и испуганно огляделась.
	
	— Не бойся, — ласково приговаривала Марпида. — Не бойся. Мы тебе плохого не сделаем. Побудь у нас, отдохни... Немного согрейся... здесь, всё-таки, не дует. — И она нежно гладила птицу.
	
	— Верно, изголодалась, — заметил слепой и протянул свой, ещё не тронутый кусок хлеба нежданной гостье.
	
	— Подкрепись немного, отдохни.
	
	Сова слегка повернула голову, увидела хлеб и стала есть. Вскоре от заветного ломтя не осталось ни крошки.
	
	Сова с благодарностью подняла глаза на своих спасителей, посмотрела вокруг, увидела голые стены и лавки, потухшую печь, худые лица слепого и девушки и поняла, что они отдали свой последний кусок хлеба.
	
	— Вы... добрые, честные... — промолвила Сова, и, услышав, что птица говорит по-человечески, люди вздрогнули от неожиданности. — Спасибо вам!
	
	Она поднялась, попыталась сделать пару шагов, но пошатнулась и застонала. Марпида заметила рану на левом крыле птицы, схватила полотенце, которое вышивала для Перы, и подошла к Сове.
	
	— Дай, я перевяжу.
	
	Сова медленно приподняла крыло, и Марпида увидела под ним толстую книгу. На её золотистом переплете темнели большие пятна.
	
	— Это моя кровь, — поймав вопросительный взгляд девушки, спокойно сказала Сова. — Сотри её, пожалуйста, книги нужно держать в чистоте
	
	Марпида бережно перевязала рану и вытерла переплёт.
	
	Сова приободрилась и села на лавку рядом со стариком.
	
	— Когда-то я любил читать, — вздохнул слепой. — а теперь не могу.
	
	— Книга мудрости доступна и тебе, — заметила Сова.
	
	— Книга мудрости?! — старик и Марпида были ошеломлены.
	
	— Я догадываюсь о твоём заветном желании, — продолжала Сова, обращаясь к слепому, — и постараюсь тебе помочь. Девушка, — торжественно произнесла птица, спускаясь на пол. — Осторожно возьми у меня из-под крыла Книгу мудрости, положи мне на голову и разверни на... сто двенадцатой странице. А ты, старик, протяни руку, внимательно ощупывай каждую букву, каждое слово — и тебе многое откроется.
	
	Взволнованный дед положил пальцы на выпуклые буквы чудесной Книги и медленно стал их разбирать на ощупь, складывая в слова:
	
	
	
	
\qquad \qquad \qquad 	Только сильный путь осилит,
	
\qquad \qquad \qquad 	Только смелый не отступит.
	
	
	
	
	Видимо, эти строки повторялись на каждом листе Книги мудрости.
	
	
	
	
\qquad \qquad \qquad 	Только тот, кто ради многих
	
\qquad \qquad \qquad 	Будет с милою в разлуке,
	
\qquad \qquad \qquad 	Только тот, кто с добрым сердцем.—
	
\qquad \qquad \qquad 	Вновь глаза твои откроет.
	
	
	
	
	Сова указала как раз ту страницу, где говорилось о волшебных свойствах Камня жизни, о том, как вернуть потерянное зрение. Пальцы деда Ивана задрожали, когда внизу страницы он разобрал долгожданные слова: 
	
	
	
	
\qquad \qquad \qquad 	Чтобы зрение вернулось,
	
\qquad \qquad \qquad 	Ветку рога от оленя.
	
\qquad \qquad \qquad 	Что служил у Хановея…
	
	
	
	
	Слепой осторожно передвинул пальцы на соседнюю страницу и продолжал:
	
	
	
	
\qquad \qquad \qquad 	В тундру белую направит
	
\qquad \qquad \qquad 	Точно по звезде Полярной.
	
	
	
	
	Эти слова никак не вязались с предыдущими. Далее снова говорилось о Камне жизни... Старик умолк в недоумении; потом вновь прочел вслух написанное.
	
	Сова удивлённо спросила: перевернул ли он страницу? Нет. Книга открыта. Он читает подряд, перешёл с одного листа на другой, но не переворачивал.
	
	Сова опустила Книгу на пол, склонилась над ней и растерянно заморгала глазами.
	
	— Беда. — прошептала она, — страшное несчастье: из Книги мудрости вырван лист.
	
	— Что же делать? — разом спросили дед Иван и Марпида.
	
	— Отыскать во что бы то ни стало! — Сова вновь и вновь переворачивала и пересчитывала листы. Сомнений не было: одного не хватало.
	
	Занимался рассвет. Вьюга улеглась. Мариида выбежала из дома, подошла к тому месту, где нашла Белую Сову; стала разгребать снег, осмотрела двор, всё вокруг избы, но ничего похожего на утерянный лист не обнаружила.
	
	Следом вышла расстроенная Сова; тоже всё оглядела и, достав из-под снега небольшой камешек, вернулась в избу и опустила его на лавку, на мох; потом трижды стукнула клювом по камешку. Брызнули искры. Сухой мох задымился. Марпида сразу сунула его в печку, сверху положила несколько щепок, и вскоре разгорелся яркий огонь. В избе потеплело.
	
	— Вот все, что я пока могу для вас сделать, — грустно сказала Сова. — Прощайте. Мне пора.
	
	Напрасно дед Иван и Марпида уговаривали гостью остаться. Исчезновение листа из Книги мудрости не давало ей покоя. Хранительница догадывалась, что могла его потерять только при стычке с Вэрсой и Бабой-Ёмой, и задумала отправиться на место поединка.
	
	Если пропавший лист не отыщется, Сова решила полететь на вершину Тэв-Поз-Из и там дождаться хозяев, чтобы понести наказание за свою оплошность.
	
	Пища и отдых восстановили силы Совы. Она еще раз поблагодарила своих спасителей и снова отправилась в путь.
	
	— Улетела Белая Сова. — грустно сказала Марнида, глядя вслед быстро удалявшейся птице.
	
	— И улетела надежда. — добавил слепой и потупил голову.
	
	Утро выдалось морозное и ясное. Сова высоко поднялась над лесом, легко отыскала дорогу и вскоре очутилась над уже знакомой нам поляной. Птица сделала над нею несколько кругов и, когда убедилась, что ей никто не угрожает, опустилась на землю.
	
	Хранительница поняла, что после её отлета здесь происходили бурные события. Она наткнулась на застывший труп Вэрсы, обнаружила остатки костра и следы ожесточенной борьбы. Но листа из Книги мудрости не было. Возможно, его кто-то подобрал, а то и сам Хановей в своём диком разгуле занес невесть куда, и искать его бесполезно.
	
	Расстроенная неудачными поисками, Сова полетела прочь от злосчастного места. Вскоре на горизонте показались густые цепи гор. Над ними гордо поднималась скрытая облаками вершина Тэв-Поз-Из. Подлетев к ней, Хранительница Книги мудрости уселась на крутом уступе скалы, ожидая появления своих грозных хозяев.
	
	
			\section[14. На уступе Тэв-Ноз-Из]{\center \textcolor{red}{14. НА УСТУПЕ ТЭВ-НОЗ-ИЗ}}
	
	
\lettrine[findent=0pt]{\textbf{\textcolor{red}{К}}}{}аменным гнездом ветров — по-коми Тэв-Поз-Из — называлась эта скалистая гора. С неё в разные стороны срывались буйные вихри и ураганы, разлетались облака и снежинки. Но уступ у самой вершины, на котором сидела Сова, представлял собой маленькую площадку, с трёх сторон ограждённую от ветров каменной скалой. Под ним зияла глубокая пропасть.
	
	Вокруг завывали ветры, в тёмные ущелья низвергались лавины.
	
	Время от времени наступало затишье. Белые пушистые облака останавливались у ног Совы, и тогда казалось, что она неотделима от них и, подобно мраморному памятнику, возвышается над застывшим сказочным морем. А потом облака медленно трогались с места, редели, в них появлялись просветы и далеко внизу вырисовывались обледенелые склоны гор. Чуть ниже ледников чернели кустарники, а у подножия из тумана, словно пики, торчали верхушки деревьев. Когда рассеивались облака и туманы, открывались дальние дали, одетые зелёными лесами, прорезанные бесчисленными ручьями и реками, позолоченные солнцем. Это была земля Коми.
	
	В своем неприступном убежище Сова безбоязненно положила перед собой Книгу мудрости, опять проверила и пересчитала листы. Увы, одного не хватало.
	
	Всё приближающийся разбойничий свист вывел Сову из задумчивости. В роскошной упряжке бородач Хановей объезжал горы. Когда его олени поравнялись с уступом, Сова окликнула гуляку. Но тот не остановился. Внизу он приметил каких-то путников и торопился потехи ради хорошенько обдуть их своим морозным дыханием, запорошить им глаза снегом, сбить с пути, чтобы после со смехом столкнуть несчастных в какое-нибудь ущелье.
	
	Хановею было не до Совы. Он услышал её зов, но даже не оглянулся, только ещё громче засвистел, совсем уж по-хулигански, как глупый уличный мальчишка, и вскоре его белые олени, серебряные нарты и он сам исчезли в искрящейся снежной пыли.
	
	Сова только покачала головой вслед: что с него спросишь? Ветер…
	
	Тяжело вздохнув, она склонилась над открытой книгой. Но читать не пришлось. Книга захлопнулась, и знакомый хрипловатый голос Пурги ехидно процедил над ухом птицы:
	
	— Что? Литературой занимаешься? Учишься? — и Пурга мелкой дрожью затряслась от смеха, очень довольная своей грубой шуткой; уселась на острую вершину скалы так, что нечесаные седые волосы свесились над уступом в самую пропасть, и пронзительно посмотрела на молчавшую Сову:
	
	— Что-то случилось?
	
	— Да. Книгу мудрости запечатывать нельзя.
	
	Услышав о пропаже листа, Пурга пришла в ярость и срочно вызвала Хановея: исчез именно тот лист, в котором говорилось о Камне жизни. Даже бестолковый Хановей страшно разволновался, когда сестрица напомнила, что с помощью волшебного Камня их могут лишить власти и вообще уничтожить.
	
	Не зная, в чьи руки попала утерянная страница и где её искать, злые силы рассудили, что нужно немедленно отправиться в тундру, к тому месту, где зарыт Камень жизни, и устроить засаду. Ведь тот, кто завладел листом из Книги мудрости, непременно придёт в тундру за Камнем, и тут-то они его подкараулят. Второпях Пурга и Хановей не поинтересовались, на каких именно строчках обрывалась предыдущая страница, с каких слов начиналась следующая после утерянной. Они поспешили к холмам Великой северной равнины, чтобы добраться туда раньше обладателя листа из Книги мудрости. В наказание за потерю злые силы тут же приковали Хранительницу Книги ледяными оковами к скале Тэв-Поз-Из. Оленью упряжку Хановея они оставили на уступе, а сами налегке, чтобы быстрее примчаться к цели, взмахнули вьюжными крыльями и стремительно понеслись на север.
	
	Сова печально опустила голову. Прикованная к скале, она не сможет добыть себе пищу. Хозяева обрекли её на голодную смерть. Да, она виновата: поверила Бабе-Ёме, польстилась на обещанных мышей и подпустила слишком близко колдунью и Вэрсу, из-за чего пропал злополучный лист. Но разве она вот уже почти двести лет не оберегала Книгу, переданную ей предыдущей Хранительницей? Разве она не делала всё, что могла для спасения Книги и всех страниц? Разве не старалась найти пропавший лист и честно, а могла бы утаить (!), не сказала о нем своим хозяевам? А что они сами предприняли для спасения Книги? Почему они безнаказанно делают глупости и гадости, а других, менее виновных, осуждают и так жестоко карают? Подумав, Сова заключила, что хозяева настолько несправедливо поступили с ней, что она, в свою очередь, больше им служить не обязана. А Хановей и Пурга вскоре потеряют свою власть: лишняя жестокость — признак слабости, неуверенности, даже страха. Сделав такой вывод, мудрая птица успокоилась и поглядела на белых оленей Хановея. Они стояли неподвижно возле неё, держа прямо гордые головы, увенчанные ветвистыми рогами. Покрытые инеем, в лунном свете они отливали серебром. Сова невольно залюбовалась красивыми животными, но не сказала ни слова, зная, что олени неразговорчивы.
	
	Отогнав печальные мысли и рассудив, что и те двести лет, которые она прожила, — это вовсе немало, — птица закрыла глаза и задремала. Но сон продолжался недолго. Его прервало тревожное постукивание копыт.
	
	Сова открыла глаза. Олени, вытянув шеи, смотрели на край уступа. Над ним показалась сперва рука, затем — человеческая голова. Сова хотела было улететь, но вспомнила, что прикована. Звеня ледяными цепями, она сделала два шага к краю уступа и уже хотела ударом клюва сбросить в пропасть отчаянного пришельца, но на его открытом лице не прочла ни страха, ни злых намерений. Смельчак поднялся на уступ и почтительно приветствовал птицу.
	
	— Кто ты и зачем сюда пришел? — спросила она.
	
	— Я охотник Пера, сын Кудым-Оша, — ответил незнакомец. — Я пришел сюда, потому что хочу добыть счастье моим землякам и всем честным людям, хочу, чтобы они жили в тепле, не знали нужды, голода и холода. Я хочу…
	
	Сова перебила его:
	
	— Ты хочешь найти Камень жизни?
	
	— Да, — кивнул Пера. — Но как попало к тебе это вышитое полотенце? — Он указал на перевязанное крыло птицы.
	
	Сове сразу понравился этот плечистый охотник и она рассказала ему о Марпиде, о слепом, о их беде и о листе, вырванном из Книги мудрости.
	
	Пера достал лист из-за пазухи.
	
	— Этот? Возьми.
	
	О, как обрадовалась Белая Сова! Охотник разбил её ледяные оковы, и она затанцевала на месте от счастья, когда вложила в Книгу полученный лист. Внимательно просмотрев его, она захлопала крыльями и расхохоталась.
	
	— Ох-хо-хо! — смеялась она. Ох-хо-хо! Я совсем забыла: на утерянном листе говорится только о свойствах волшебного Камня, а как его найти, написано на следующем. А они, — Сова кивнула в сторону своих улетевших хозяев, — так всполошились:..
	
	— Читай! — торжественно сказала Хранительница, раскрывая перед богатырем Книгу мудрости. — Вот путь к Камню жизни.
	
\qquad \qquad \qquad 	Он откроется тому лишь
	
\qquad \qquad \qquad 	С Тэв-Поз-Из крутой вершины, — читал Пера.—
	
\qquad \qquad \qquad 	Кто оленей Хановея
	
\qquad \qquad \qquad 	В тундру белую направит
	
\qquad \qquad \qquad 	Точно по звезде Полярной.
	
	Далее охотник вычитал, как с помощью одного из оленей Хановея найти в тундре место, где спрятан чудесный Камень.
	
	Тут Пера вопросительно посмотрел на Сову: где найти этого оленя?
	
	Сова подошла к упряжке и остановилась перед красавцем-оленем с маленьким темным пятнышком на лбу:
	
	— Вот он, — сказала Сова.
	
	Пера достал из пестеря, где была охотничья добыча, мох-ягель, который так любят олени, погладил их и начал кормить. Животные хорошо помнили грубое обращение Хановея и стали ласкаться к богатырю. Однако нужно было поскорее помочь старику и Марпиде и в то же время поспеть в тундру раньше Хановея и Пурги.
	
	Посоветовавшись, решили, что охотник помчится в тундру, а Сова — к избушке и передаст Марпиде и слепому сумку с дичью.
	
	Увлеченные разговором, Пера и Сова не обратили внимания на странное беспокойство оленей. Они напряженно смотрели вниз, угрожающе выставив рога над пропастью. Там, скрытая уступом скалы, затаив дыхание, стояла Баба-Ёма и подслушивала. С большим трудом добралась она сюда. Ей хотелось метнуть в Перу копьё с серебряным наконечником, но мешал уступ, а взобраться на него было нелегко.
	
	Переведя дух, старуха вцепилась длинными ногтями в обледенелые края уступа и уже хотела вскарабкаться на него, чтобы поразить охотника смертоносным оружием, как вдруг увидела у самого лица направленные на неё грозные рога оленей. Злодейка невольно отпрянула назад, руки её соскользнули е уступа, и она с криком полетела в пропасть. Когда Пера и Сова подскочили к её краю, внизу ничего не было видно: все заволокло снежной пылью.
	
	— Ёма, — качнула головой догадливая Сова. — Так ей и надо.
	
	Птица взмахнула крыльями и полетела к избушке.
	
	Пера сел в серебряные нарты, указал длинным хореем в сторону Полярной звезды, на север, и громко крикнул. В то же мгновение он почувствовал, как нарты отделились от земли, и волшебные олени понесли его через горные отроги и ущелья, через леса и реки к Великой северной равнине.
	
	Дорогие мои читатели! Знаю, что все ваши симпатии на стороне Перы и его друзей. Поверьте, что я не меньше, чем вы, желаю им добра. И, хотя я человек не злой, а всё-таки вздохнул с облегчением, когда коварная Ёма не сумела пронзить копьём смелого охотника и свалилась в пропасть. Я даже подумал, что она погибла, и, ей-ей, не огорчился. Кто не жалеет и не любит других — сам не заслуживает любви и его самого не стоит жалеть. Не пожалел я и Бабы-Ёмы. Но... против моей воли, она не погибла: сорвалась с огромной высоты, но упала в глубокий мягкий снег, такой глубокий, что погрузилась в него с головой и даже не ушиблась, только испугалась; потеряла головной убор, горшок. Он свалился с её вздыбленных от страха волос и утонул в снегу. Но падение ещё больше рассердило мстительную старуху. Теперь, зная планы наших друзей, она задумала погубить их всех, но не сразу, а постепенно, по очереди. Она понимала, что без Книги мудрости от Камня жизни мало проку, а без него — и Книге не та цена... Значит, силен будет лишь тот, кто сумеет добыть и Книгу и Камень. Чем же завладеть раньше?..
	
	Ёма долго барахталась в рыхлом снегу, прежде чем сумела выкарабкаться со дна ущелья. Выбравшись из него, она заковыляла к лесу. Тут ей повезло: навстречу бежал матерый рыжий волк. Колдунья окликнула его. Он остановился. Старуха села на него верхом, что-то шепнула на ухо, и хищник быстрее лошади побежал через густой лес.
	
				\section[15. Луч волшебного камня]{\center \textcolor{red}{15. ЛУЧ ВОЛШЕБНОГО КАМНЯ}}
	

	\lettrine[findent=0pt]{\textbf{\textcolor{red}{Т}}}{}еперь нам пора перенестись В самые дальние края и посмотреть, что делается на просторах Великой северной равнины, в так называемой тундре? Летом солнце, не смыкая глаз, любовалось её просторами. Ночь не опускалась на землю. Едва сходил снег, как поднимались травы, торопились зазеленеть листочки на кустах и скрюченных карликовых ивах и берёзках; распускались цветы, очень яркие, но... совсем без запаха, так как его не любили властители Великой северной равнины, Хановей и Пурга. Однако, не думая о них, у бесчисленных озер шумели залётные птицы, передавали друг другу всякие волнующие новости, выводили пискливых пушистых птенцов. Увы, веселая пора продолжалась недолго. Темные тучи закрывали солнце, и оно в слезах покидало свои любимые края. Залётные птицы отправлялись вслед за ним, в жаркие страны. Только-только распустившиеся цветы обламывал ветер. Мороз сковывал землю. Воцарялась долгая ночь. Лишь звезды и волшебное ожерелье, заброшенные когда-то высоко-высоко могучей рукой древнего чудо-богатыря, покачивались в небе, озаряя печальным светом снежные просторы.
	
	Сюда спешили Пурга и Хановей. Он помогал ей лететь, поддерживал её крылья и время от времени дул в них, как в паруса.
	
	Взметая снежные вихри, властители северной равнины опустились на один из холмов.
	
	Вокруг, насколько хватал глаз, виднелись точно такие же, покрытые снегом холмы.
	
	Пурга и Хановей растерялись: где караулить? Под каким холмом Камень жизни? Какой холм стеречь?
	
	Властители закружили по равнине, напрасно пытаясь припомнить, где зарыт чудесный Камень. Потом они забились в глубокий овраг и задумались.
	
	Пурга посоветовала брату слетать на Тэв-Поз-Из и узнать по Книге мудрости, где зарыт чудесный Камень, но тут же вспомнила, что страница, на которой об этом написано, утеряна. Тем не менее, жестокая властительница велела брату вернуться на Тэв-Поз-Из, убить Белую Сову, взять Книгу, принести в тундру, закопать в мерзлую землю и завеять снегом.
	
	Хановей согласился, что все это нужно сделать, но не хотел возвращаться в такую даль. Он предложил полететь Пурге. Та отказалась, ссылаясь на нездоровье, на свой возраст и на то, что она всё-таки существо женского рода, а женщинам нужно уступать.
	
	Брат доказывал, что он не моложе сестры, а если принять во внимание его длинную седую бороду, то даже старше. А стариков следует уважать. Спор закончился тем, что бросили жребий. Пурга взяла в свои холодные руки две сосульки, длинную и короткую. Кто вытянет длинную — тому лететь. Она спрятала руки за спиной, спросила брата, какую он выбирает, и незаметно подсунула ему длинную сосульку.
	
	Хановей выругался, предупредил усталую сестру, чтоб зорко смотрела по сторонам, шумно вздохнул и полетел прочь.
	
	Едва он скрылся, как Пурга расхохоталась. Ей очень понравилось, как она обманула брата. Потом она зевнула, поглядела по сторонам и, не заметив ничего подозрительного, намела себе высокий снежный чум, зарылась в него и громко захрапела.
	
	Когда обманщица заснула, на небе показались звезды, уронили искорки лучей на снега, и на них сверкнули серебряные блестки. В вышине появились разноцветные факелы, Они медленно передвигались вдали, над горизонтом. Это были сполохи северного сияния, отблески чудесного ожерелья. Дрожащие блики падали на равнину. Вся она озарилась таинственным голубоватым светом.
	
	Но Пурга не видела всей этой красоты. Она крепко спала и не слышала усиливающегося посвиста полозьев, не видела, как на горизонте показалось белое облачко, и, все увеличиваясь, стало быстро приближаться: это олени Хановея везли Перу.
	
	Невдалеке от снежного чума Пурги, на холме у обрывистого берега замерзшей реки, упряжка остановилась. Белый олень с тёмным пятном на лбу трижды ударил копытами по холму. Холм раскололся. Чёрная бездна раскрылась у ног Перы. Он заглянул в неё и в глубине увидел красноватый свет, как от тлеющего уголька.
	
	Пера опоясал себя одним концом длинной веревки, другой — привязал к рогам оленя, неподвижно стоявшего над пропастью, и, цепляясь за её стены, начал медленно спускаться.
	
	По мере спуска становилось все теплее и светлее. На дне пропасти от жары перехватывало дыхание. Там сиял раскаленный краешек бесценного Камня жизни. Пера охотничьим ножом очистил его от налипшей земли. Помня советы Книги мудрости, богатырь оставил одну сторону Камня неочищенной, чтобы его можно было держать в руках.
	
	К удивлению Перы загадочный Камень оказался вовсе не тяжёлым. С ним нетрудно было взбираться по верёвке. Но ещё не успел обрадованный богатырь подняться наверх, как небо внезапно потемнело. Крупные хлопья снега полетели в лицо смельчаку. Едва он ухватился рукой за один край обрыва, как с другого края в пропасть рухнула огромная ледяная глыба.
	
	Пока Пера доставал Камень жизни, в тундру вернулся Хановей. Не найдя Белую Сову на вершине Тэв-Поз-Из, бородач помчался к сестре.
	
	Он увидел снежный чум, неподалеку — расколотый холм и своих оленей и всё понял. В ярости он несколько раз ударил оленей хореем, но они не шелохнулись, затем диким окриком разбудил Пургу и ткнул её носом в расселину холма.
	
	Злые силы, стараясь погубить Перу, заволокли небо тучами, погасили звёзды и стали забрасывать пропасть, из которой выбирался охотник, снегом и льдом. Но злые силы опоздали. Красный луч ударил им навстречу. Они завыли, отпрянули в сторону и попытались обрушить на Перу глыбы мерзлой земли. Но он уже успел выбраться наверх, поднялся во весь рост и, держа над головой Камень жизни, поворачивал его острым краем во все стороны, и куда падал его луч, там снег моментально исчезал, а из-под земли выбивалась нежная зелёная трава.
	
	Напрасно Пурга и Хановей пытались спрятаться от губительного для них красного луча. Он всё-таки скользнул по их бледным лицам. С коротким воплем злые силы беспомощно рухнули на землю грудами снега и тут же растаяли. На том месте, где они стояли, образовалось два мутных озера.
	
	Над тундрой показалось солнце. Необъятная равнина проснулась, зажурчала ручьями, зазеленела молодой травой.
	
	Пера расстегнул ворот рубахи, вытер вспотевший лоб и присел на нарты. Только сейчас богатырь почувствовал усталость. Он сидел, глядя на мирно пасущихся оленей, на зверей и птиц, безбоязненно бегавших вокруг, и веки его тяжелели, смыкались. Сладкая дрёма клонила голову к нартам.
	
	Но отдохнуть не пришлось. Дремоту разогнал крик белой Совы. Встревоженно махая крыльями, она появилась над ним. На её шее беспомощно болтался обрывок ремешка от Книги мудрости. То, что в нескольких словах сообщила птица, сразу развеяло сон. Пера вскочил на нарты и погнал оленей Хановея в родные края.
	
	
				\section[16. Поединок]{\center \textcolor{red}{16. ПОЕДИНОК}}


	\lettrine[findent=0pt]{\textbf{\textcolor{red}{С}}}{}меркалось, когда рыжий волк доставил Бабу-Ёму к жилищу деда Ивана и Марпиды. Старуха даже не поблагодарила зверя за оказанную услугу, и он убежал опять в чащу, подальше от колдуньи.
	
	Ёма подождала, пока окончательно не стемнело, — а ночью она видела не хуже, чем днем, — подошла к избе и заглянула внутрь.
	
	Слепой и Марпида, очень похудевшие, но бодрые, сидели за столом при свете лучины и перед ними лежала открытая Книга мудрости. Её Хранительница, Белая Сова, в углу у печки преспокойно чистила клювом свои пёрышки.
	
	При виде этой мирной картины колдунья чуть не задохнулась от злости, но сдержалась. Отойдя в сторонку, она достала из пестеря две верёвки и снова вернулась к домику. Его обитатели вскоре легли спать. Когда они крепко уснули, Ёма тихонько произнесла заклинание. В стене открылся большой проём, и злодейка бесшумно проникла в избу.
	
	Накинуть приготовленную петлю на Сову не удалось: чуткая птица проснулась и подала сигнал тревоги. Но старик и девушка не услышали, и злодейка их быстро связала.
	
	Сова попыталась вступить в единоборство с Ёмой, но та бросилась на птицу с большим охотничьим ножом. Хранительнице едва удалось уклониться от смертоносного взмаха. Нож рассёк ремешок, на котором висела Книга мудрости. Книга упала на пол. Ёма наступила на неё ногой, продолжая размахивать ножом, чтобы не подпустить птицу. Но та вдруг исчезла в проёме стены, который старуха ещё не закрыла за собой.
	
	Перед колдуньей лежала заветная Книга мудрости, а прочесть её старуха не могла: поленилась когда-то выучиться грамоте. Теперь Баба-Ёма ругала себя за это, да поздно... Вот и хотела она заставить пленников открыть ей тайны великой Книги, чтобы стать еще могущественнее и принести добрым людям еще больше зла.
	
	Колдунья залила в печке огонь, уселась у стола и сладким голосом начала убеждать прочесть ей несколько страниц вслух. Но пленники отказались, Марпида сказала, что сейчас темно и букв не видно, а старик заметил, что его связанные руки болят и так отекли от верёвок, что он не сможет нащупать буквы.
	
	Слепой и девушка надеялись, что скоро вернётся Пера и выручит их из беды.
	
	Ёма догадалась об этом, но ничего не возразила. Она была уверена, что ждать его бесполезно: если он еще не погиб, то вот-вот должен погибнуть в неравной борьбе с Пургой и Хановеем, которых считала непобедимыми. Сама же она мечтала с помощью Книги мудрости каким-либо обманным путём добыть Камень, не вступая в открытую борьбу с властителями севера.
	
	Старуха положила Книгу на стол и, поглаживая костлявой рукой золотистый переплет, завела разговор о том, что опа вовсе не такая плохая, как все думают, и, если б ей открылись тайны чудесной Книги, она бы, конечно, исправилась и полюбила людей.
	
	Ни одному слову лгуньи пленники не верили. Дед Иван прямо заявил, что Книга мудрости хороша только в добрых руках.
	
	Ни щедрые обещания, ни угрозы не подействовали. Старик и Марпида не хотели вообще разговаривать с воровкой. Их руки и ноги, туго стянутые верёвками, распухли и посинели; их мучили голод, холод и жажда.
	
	Ёма ударяла кремнем об огниво, обещала жарко натопить избу и даже уйти совсем, если ей прочтут Книгу. Но пленники молчали.
	
	Тогда взбешенная колдунья решила их уничтожить и произнесла страшное заклинание.
	
	Потемнело в избе, сами распахнулись окно и дверь, огромные щели раскрылись в стенах. Со всех сторон на беззащитных пленников обрушились потоки дождя и снега; зашумела холодная буря, ворвались пронизывающие насквозь ледяные ветры.
	
	Громко расхохоталась Баба-Ёма, схватила Книгу мудрости, надеясь узнать где-нибудь в другом месте, что в ней написано; бросила торжествующий взгляд на свои жертвы, повернулась к двери И вдруг отшатнулась: на пороге стоял Пера с Камнем жизни в руках. Метнулась злодейка к окну, но в нем увидела острый клюв Белой Совы.
	
	— Ку-уда?! — угрожающе крикнула птица.
	
	Выставила Ёма перед собой Книгу, как щит, быстро произнесла колдовское слово, вдруг превратилась в лису и юркнула к порогу. Но в тот же миг ее достал острый луч волшебного Камня.
	
	Веером рассыпались рыжие искры и от Бабы-Ёмы осталось на пороге только тёмное пятно.
	
	Пера охотничьим ножом перерезал верёвки, опутывавшие пленников. Девушка не дышала.
	
	— Она так ждала тебя... — еле шевеля губами, прошептал старик.
	
	У Перы опустились руки.
	
	— Скорее положи Камень жизни на очаг! — воскликнула Сова.
	
	Охотник послушался — и сразу закрылись щели в стенах, затворились окно и дверь; изба наполнилась радужным теплым светом. Он заструился по лицам старика и Марпиды. Губы и щеки девушки порозовели, дрогнули ресницы. Она вздохнула, открыла глаза и увидела Перу.
	
	Дед Иван нежно гладил её потеплевшие руки, а Сова, сидя на подоконнике, растроганно шмыгала клювом и поспешно перелистывала Книгу мудрости.
	
	Дед подошел к птице, и она раскрыла перед ним Книгу:
	
	— Вот, нашла. Читай.
	
	Слепой стал запинаться от волнения, когда его пальцы нащупали долгожданные слова:
	
	
	
	
\qquad \qquad \qquad 	Чтобы зрение вернулось,
	
\qquad \qquad \qquad 	Ветку рога от оленя,
	
\qquad \qquad \qquad 	Что служил у Хановея,
	
\qquad \qquad \qquad 	Нужно сжечь на Камне жизни
	
\qquad \qquad \qquad 	И еще горячий пепел
	
\qquad \qquad \qquad 	Приложить к дрожащим векам.
	
	
	— Торопись! — воскликнула Белая Сова. — Камень жизни согрел не только твоё жилище. Скоро выбьется из-под земли багульник, и белые олени уйдут в тундру.
	
	Пера выбежал из избы и через минуту вернулся с маленькой веточкой оленьего рога; положил её на очаг, и когда она сгорела, бережно собрал ещё горячий пепел и осторожно приложил к векам слепца.
	
	Словно жаром охватило деда Ивана. Бодрящая сила наполнила грудь, покрыла румянцем лоб и щеки. Веки слепого затрепетали, глаза медленно открылись. Сперва будто в тумане, затем все яснее и яснее он увидел Перу и Марпиду и удивился: какая она большая. Ведь он её видел, когда она была совсем маленькой девочкой.
	
	Еще не веря в исцеление, старик протянул к ним руки.
	
	Кончилась зимняя ночь. Дрогнула земля: олени Хановея стукнули по ней серебряными копытами и помчались в тундру.
	
	— Пора и мне, — вздохнула Белая Сова. — Прощайте!
	
	Напрасно старик, Марпида и Пера убеждали птицу остаться. Она только грустно покачивала головой. Она очень полюбила наших друзей, но её ждала родина — великая северная равнина. А зов родины сильнее всего.
	
	— Мне уже двести лет, я не молода, — призналась Сова, — я многое видела и, может быть, кое-чему научилась. Но с годами и крылья, и память слабеют. Мне всё труднее оберегать и носить с собой тяжелую Книгу мудрости. Чувствую, она мне не по плечу.
	
	— Возьми её, Пера, — обратилась она к охотнику. — Пусть такие же люди, как ты, вписывают в неё новые страницы и учатся владеть Камнем жизни.
	
	С этими словами Белая Сова подала Книгу богатырю, поклонилась, вышла из избы и полетела на север.
	
	Пера, дед Иван и Марпида долго смотрели вслед птице, дивились, что так быстро пришла в их края солнечная весна, и думали о Книге мудрости и волшебном Камне жизни, которые должны помочь всем добрым и смелым людям в их борьбе за счастье, тепло и свет.
	
	\newpage
	\tableofcontents
	
	\thispagestyle{empty} % 
	
	\newpage
	
	\setcounter{secnumdepth}{0}  
	
	\phantomsection
	
\end{document}


