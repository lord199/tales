
\documentclass[oneside,final,14pt]{extreport}
%\usepackage[koi8-r]{inputenc}
\usepackage[russianb]{babel}
\usepackage{vmargin}
\setpapersize{A4}
\usepackage[T2A]{fontenc}
\usepackage[utf8x]{inputenc}  % more recent versions (at least>=2004-17-10)
%\usepackage[russian]{babel}
\setmarginsrb{2cm}{1.5cm}{1cm}{1.5cm}{0pt}{0mm}{0pt}{13mm}
\usepackage{indentfirst}
\usepackage{nicefrac} % For comparison
%\usepackage{xfrac}	% Works better with other fonts
\usepackage[unicode, pdftex]{hyperref}
\usepackage{lettrine}
\usepackage[usenames]{color}
\usepackage{colortbl}
\sloppy
\begin{document}
	
	\section*{Описание}
	
	{\bf Название:} Волшебный Камень и Книга Белой Совы
	
	{\bf Автор:} Александр Соломонович Клейн
	
	{\bf Издательство:} Сыктывкар: Коми книжное издательство, 1986
	
	{\bf Художник:} Г. С. Тонков
	
	{\bf Редактор:} А. В. Некрасов
	
	{\bf Художественный редактор:} В. Б. Осипов
	
	{\bf Технический редактор:} А. Н. Вишнева
	
	{\bf Корректоры:} А. А. Надуткина, Т. И. Попова
	
	{\bf Аннотация:} Эта повесть-сказка, написанная на основе древних коми поверий и преданий, посвящена удивительным приключениям добрых и смелых людей, их борьбе против злых сил за право владеть волшебным камнем и Книгой мудрости.
	\thispagestyle{empty} % выключаем отображение номера для этой страницы
	\newpage
	
	\begin{titlepage}
		
		\begin{center}
			%\vfill
			
			%\vfill
			\topskip0pt
			\vspace*{\fill}
			
			
			{\large\bf Александр Клейн\\}
			\ \\
			\ \\
			{\Huge\bf ВОЛШЕБНЫЙ КАМЕНЬ И КНИГА БЕЛОЙ СОВЫ\\}
			\ \\
			\ \\
			\ \\
			Повесть-сказка
			\ \\
			\ \\
			По мотивам коми фольклора
			\vspace*{\fill}    
			
			\vfill
			
			Сыктывкар
			
			Коми книжное издательство 1986
		\end{center}
		
	\end{titlepage}
	
	\thispagestyle{empty} % выключаем отображение номера для этой страницы
	
	\newpage
	
	\begin{flushleft}
		\begin{verse}
			\qquad \qquad Читатель-друг, ступай со мной\\
			\qquad \qquad Туда, где лес стоит стеной\\
			\qquad \qquad И птицы людям говорят\\
			\qquad \qquad Про вьюжный край, где спрятан клад,\\
			\qquad \qquad Где серебром блестят равнины,\\
			\qquad \qquad А в горных облачных вершинах\\
			\qquad \qquad Гнездятся ветры.\\
			\qquad \qquad\qquad\qquad\qquad\qquad Труден путь.\\
			\qquad \qquad Но не робей и рядом будь.\\
			\qquad \qquad Знай: одному — и мне беда.\\
			\qquad \qquad С тобой — согласен хоть куда;\\
			\qquad \qquad Один — тяжел я на подъём.\\
			\qquad \qquad А вместе — в сказку мы войдём!\\
		\end{verse}
	\end{flushleft}
	
	{%
		\centering
		\subsection*{\textcolor{red}{1. Как исчезло Зелёное Царство}}
	}
	
	\lettrine[findent=0pt]{\textbf{\textcolor{red}{К}}}{} аждые сто лет в Книге мудрости прибавлялся только один лист. Никто не мог объяснить, как это происходило, но с каждым столетием Книга становилась чуть толще. Она была написана не совсем обычно, говорят, даже стихами. В ней было много полезных советов. Но никто из людей её никогда не видел. А о её существовании знали давно. Вероятно, старые охотники, хорошо понимающие язык птиц, услышали о чудесной Книге от них. Птицы не умеют хранить тайны, особенно сороки и воробьи. Они крошки не клюнут без того, чтобы не пискнуть, не сказать что-нибудь на своём языке, не посплетничать.
	
	А птицы, безусловно, должны были знать о Книге мудрости, потому что её хранительницей была тоже птица — Белая полярная Сова. Вы её видели когда-нибудь? Огромная, круглоголовая, с большими жёлто-огненными глазами и загнутым книзу, наподобие крючка, острым клювом; с могучими широкими крыльями, покрытыми маленькими коричневыми черточками и крапинками. У неё очень тонкий слух и отличное чутье. Неправду говорят, будто Белая Сова хорошо видит только ночью. Нет, она великолепно видит и днем. Ведь там, где она обитает, в бескрайних северных равнинах, половину года царит день, а ночи в это время вообще нет. Ночь там только зимой.
	
	Почётная должность хранительницы Книги мудрости передавалась совами из рода в род. Никто не помнит — когда, сколько тысяч лет тому назад пра-пра-пра-пра-пра-прабабушка одной Белой Совы, — а совы минут очень долго, — вложила в золотистый переплёт из непромокаемой нерпичьей шкуры первый лист заветной Книги. Никто не видел и того, кто сделал этот чудесный переплёт. Но... так уж случилось, что когда-то, давным-давно, бессмертные властители сурового и прекрасного севера, снежная буря Пурга и злой ветер Хановей, сами не умея читать, доверили белым совам хранение Книги мудрости.
	
	Раз в сто лет Хановей и Пурга интересовались тем, что в ней написано, и Сова читала им вслух о богатствах подвластной им земли, о людях, которые живут в этих холодных краях.
	
	После чтения Сова уносила Книгу в какое-нибудь потайное место, а Хановей и Пурга от безделия снова начинали кружить по просторам, заносить снегом стойбища оленеводов, морозить и сбивать с дороги одиноких путников. Наподобие исполинских чумов вздымались сугробы, оглушительно свистел ветер, снежные вихри закрывали небо. Тяжело приходилось тому, кого настигали Хановей и Пурга среди Великой северной равнины, тундры.
	
	Не всегда она была такой безжизненной. На месте болотных кочек, ложбин и голых холмов, покрытых чахлым кустарником, когда-то буйно зеленели травы, высились могучие деревья; на сказочно красивых цветах, как на волнах, пестрея крыльями, покачивались огромные бабочки. Прозрачный воздух был напоён душистой свежестью и, казалось, даже слегка вздрагивал от птичьих трелей.
	
	А посреди этого зелёного царства искрился и сиял Камень, несравненно прекраснее и дороже всех драгоценных камней — волшебный Камень жизни. Его щедрые лучи оживляли все вокруг, несли тепло и свет, радость и счастье. И под пение птиц мирно дремали чудо-леса в лучах доброго Камня.
	
	Очень давно это было... Так давно, что ученые люди до сих пор все считают и никак не могут точно сосчитать, сколько лет прошло с тех пор. Не могут они точно назвать не то что день, а даже месяц или год, когда впервые прилетели в этот край Хановей и его сестра Пурга. Они были такие же, как сейчас, лохматые, шумливые, неряшливые. Седая косматая Пурга и тогда была старухой. Она, говорят. сразу родилась старой. Да и братец ее Хановей был не моложе.
	
	Как и откуда они залетели? Неизвестно. Во всяком случае, не из жарких мест, потому что оба терпеть не могли, не выносили ласкового тепла и света. Но увидели они окружающую красоту, услышали пение птиц и невольно замерли. Уж на что был груб и неотёсан Хановей, и тот не выдержал: глубоко вздохнул, залюбовался и на его сосулистую бороду, дрожа, закапали слезинки умиления.
	
	Но Пурга быстро вывела брата из оцепенения.
	
	— Чего уставился, бородач? — прохрипела она.— С твоих усов скоро ручьи потекут и мокрое место от тебя останется, а ты рот разинул. Ведь этакая жара нас вот-вот погубит. Я задыхаюсь от запаха цветов и глохну от птичьего гомона. Пора положить конец безобразию.
	
	
\end{document}




